\documentclass[12pt,final,notitlepage,onecolumn,german]{article}%
\usepackage{amssymb}
\usepackage{amsmath}
\usepackage{amsthm}
\usepackage{framed}
\usepackage{comment}
\usepackage{color}
\usepackage{graphicx}
\usepackage[breaklinks=True]{hyperref}
%TCIDATA{OutputFilter=latex2.dll}
%TCIDATA{Version=5.50.0.2960}
%TCIDATA{LastRevised=Monday, March 14, 2022 11:01:32}
%TCIDATA{SuppressPackageManagement}
%TCIDATA{<META NAME="GraphicsSave" CONTENT="32">}
%TCIDATA{<META NAME="SaveForMode" CONTENT="1">}
%TCIDATA{BibliographyScheme=Manual}
%TCIDATA{Language=American English}
%BeginMSIPreambleData
\providecommand{\U}[1]{\protect\rule{.1in}{.1in}}
%EndMSIPreambleData
\newcounter{exer}
\theoremstyle{definition}
\newtheorem{theo}{Satz}[section]
\newenvironment{theorem}[1][]
{\begin{theo}[#1]\begin{leftbar}}
{\end{leftbar}\end{theo}}
\newtheorem{lem}[theo]{Lemma}
\newenvironment{lemma}[1][]
{\begin{lem}[#1]\begin{leftbar}}
{\end{leftbar}\end{lem}}
\newtheorem{prop}[theo]{Proposition}
\newenvironment{proposition}[1][]
{\begin{prop}[#1]\begin{leftbar}}
{\end{leftbar}\end{prop}}
\newtheorem{defi}[theo]{Definition}
\newenvironment{definition}[1][]
{\begin{defi}[#1]\begin{leftbar}}
{\end{leftbar}\end{defi}}
\newtheorem{remk}[theo]{Bemerkung}
\newenvironment{remark}[1][]
{\begin{remk}[#1]\begin{leftbar}}
{\end{leftbar}\end{remk}}
\newtheorem{coro}[theo]{Folgerung}
\newenvironment{corollary}[1][]
{\begin{coro}[#1]\begin{leftbar}}
{\end{leftbar}\end{coro}}
\newtheorem{conv}[theo]{Konvention}
\newenvironment{convention}[1][]
{\begin{conv}[#1]\begin{leftbar}}
{\end{leftbar}\end{conv}}
\newtheorem{quest}[theo]{Frage}
\newenvironment{question}[1][]
{\begin{quest}[#1]\begin{leftbar}}
{\end{leftbar}\end{quest}}
\newtheorem{warn}[theo]{Warnung}
\newenvironment{warning}[1][]
{\begin{warn}[#1]\begin{leftbar}}
{\end{leftbar}\end{warn}}
\newtheorem{conj}[theo]{Conjecture}
\newenvironment{conjecture}[1][]
{\begin{conj}[#1]\begin{leftbar}}
{\end{leftbar}\end{conj}}
\newtheorem{exam}[theo]{Example}
\newenvironment{example}[1][]
{\begin{exam}[#1]\begin{leftbar}}
{\end{leftbar}\end{exam}}
\newtheorem{exmp}[exer]{Aufgabe}
\newenvironment{exercise}[1][]
{\begin{exmp}[#1]\begin{leftbar}}
{\end{leftbar}\end{exmp}}
\newenvironment{statement}{\begin{quote}}{\end{quote}}
\newenvironment{fineprint}{\begin{small}}{\end{small}}
\iffalse
\newenvironment{proof}[1][Proof]{\noindent\textbf{#1.} }{\ \rule{0.5em}{0.5em}}
\newenvironment{convention}[1][Convention]{\noindent\textbf{#1.} }{\ \rule{0.5em}{0.5em}}
\newenvironment{question}[1][Question]{\noindent\textbf{#1.} }{\ \rule{0.5em}{0.5em}}
\newenvironment{aufgabe}[1][Aufgabe]{\noindent\textbf{#1.} }{\ \rule{0.5em}{0.5em}}
\fi
\renewcommand{\figurename}{Fig.}
\let\sumnonlimits\sum
\let\prodnonlimits\prod
\let\cupnonlimits\bigcup
\let\capnonlimits\bigcap
\renewcommand{\sum}{\sumnonlimits\limits}
\renewcommand{\prod}{\prodnonlimits\limits}
\renewcommand{\bigcup}{\cupnonlimits\limits}
\renewcommand{\bigcap}{\capnonlimits\limits}
\hoffset=-1cm
\voffset=-1cm
\setlength\textheight{21.5cm}
\setlength\textwidth{14cm}
\newtheoremstyle{plainsl}
{8pt plus 2pt minus 4pt}
{8pt plus 2pt minus 4pt}
{\slshape}
{0pt}
{\bfseries}
{.}
{5pt plus 1pt minus 1pt}
{}
\theoremstyle{plainsl}
\begin{document}

\title{AIMO 2022 BH1 Geometrie: halbgare Ausarbeitung}
\author{Darij Grinberg}
\date{\today}
\maketitle

\section{Zwei Dreiecke I}

\subsection{Ceva}

Ein Dreieck $ABC$ hei\ss t \textit{entartet}, wenn seine Ecken $A$, $B$ und
$C$ auf einer Geraden liegen. Entartete Dreiecke werden meistens \"{u}berhaupt
nicht als Dreiecke angesehen; jedoch will man hin und wieder gewisse
Eigenschaften von Dreiecken auf sie anwenden, und dies ist manchmal
zul\"{a}ssig und manchmal nicht. Wir werden also im Folgenden zumindest bei
allen S\"{a}tzen explizit darauf hinweisen, ob unsere Dreiecke entartet sein
d\"{u}rfen oder nicht.

Der folgende Satz ist wohlbekannt:

\begin{theorem}
[Satz von Ceva]\label{thm.ceva}Sei $ABC$ ein nicht-entartetes Dreieck. Seien
$A^{\prime}$, $B^{\prime}$ und $C^{\prime}$ drei Punkte jeweils auf den
Geraden $BC$, $CA$ bzw. $AB$. Genau dann schneiden sich die Geraden
$AA^{\prime}$, $BB^{\prime}$ und $CC^{\prime}$ in einem Punkt, wenn%
\[
\dfrac{BA^{\prime}}{A^{\prime}C}\cdot\dfrac{CB^{\prime}}{B^{\prime}A}%
\cdot\dfrac{AC^{\prime}}{C^{\prime}B}=1
\]
gilt. (Siehe Fig. \ref{fig.1}.)
\end{theorem}

%

%TCIMACRO{\FRAME{ftbpFU}{4.1909in}{4.1435in}{0pt}{\Qcb{Ein Dreieck mit drei
%konkurrenten Ecktransversalen}}{\Qlb{fig.1}}{Figure}%
%{\special{ language "Scientific Word";  type "GRAPHIC";
%maintain-aspect-ratio TRUE;  display "USEDEF";  valid_file "T";
%width 4.1909in;  height 4.1435in;  depth 0pt;  original-width 3.9131in;
%original-height 3.869in;  cropleft "0";  croptop "1";  cropright "1";
%cropbottom "0";  tempfilename 'Ceva1.wmf';tempfile-properties "XNPR";}} }%
%BeginExpansion
\begin{figure}[ptb]%
\centering
\includegraphics[
natheight=3.869000in,
natwidth=3.913100in,
height=4.1435in,
width=4.1909in
]%
{Ceva1.wmf}%
\caption{Ein Dreieck mit drei konkurrenten Ecktransversalen}%
\label{fig.1}%
\end{figure}
%EndExpansion


Wichtige Anmerkungen zu diesem Satz:

\begin{itemize}
\item Die Strecken sind hier und auch im Folgenden orientiert -- d.h., wir
w\"{a}hlen auf jeder der Geraden $BC$, $CA$ und $AB$ eine Richtung (egal wie),
und messen eine Strecke $XY$ positiv wenn $Y$ hinter $X$ in dieser Richtung
liegt und sonst negativ.

\item Mit \textquotedblleft schneiden sich in einem Punkt\textquotedblright%
\ meine ich \textquotedblleft schneiden sich in einem Punkt oder sind
parallel\textquotedblright. Das hei\ss t, ich arbeite in der projektiven
Ebene. Wenn also mehrere Geraden zueinander parallel sind, dann schneiden sie
sich in einem unendlich fernen Punkt. Wenn man im Satz von Ceva wissen will,
ob sie parallel sind oder nicht, hilft der Strahlensatz:

\item In entarteten F\"{a}llen der Sorte $A^{\prime}=B$ bleibt der Satz
richtig ($0$ und $\infty$ sind beide nicht $1$). Im besonders entarteten Fall
$A^{\prime}=B^{\prime}=C$ wird er sinnlos ($0\cdot\infty$). Im Fall, wenn die
Punkte $A$, $B$ und $C$ auf einer Geraden liegen, ist er leider falsch -- die
Geraden sind immer parallel, aber das Verh\"{a}ltnisprodukt muss nicht $1$
sein. Daher \textquotedblleft nicht-entartetes Dreieck\textquotedblright.

\item Ein paar gebr\"{a}uchliche aber v\"{o}llig optionale Notationen: Geraden
wie $AA^{\prime}$, $BB^{\prime}$ und $CC^{\prime}$ in Satz \ref{thm.ceva}
hei\ss en \textit{Ecktransversalen} (\textquotedblleft
cevians\textquotedblright\ im Englischen); das Dreieck $A^{\prime}B^{\prime
}C^{\prime}$ hei\ss t das \textit{Cevadreieck} des Schnittpunktes von
$AA^{\prime}$, $BB^{\prime}$ und $CC^{\prime}$ (falls dieser Schnittpunkt
existiert); drei Geraden, die sich in einem Punkt schneiden (oder parallel
sind), hei\ss en \textit{kopunktal} oder \textit{konpunktal} oder
\textit{konkurrent}.
\end{itemize}

Beweisidee zu Satz \ref{thm.ceva}: F\"{u}r die $\Longrightarrow$-Richtung kann
man eine Parallele zu $BC$ durch $A$ zeichnen und alle
Streckenverh\"{a}ltnisse auf sie projizieren (mithilfe des Strahlensatzes).
Den Fall $AA^{\prime}\parallel BB^{\prime}\parallel CC^{\prime}$ sollte man
eigentlich separat betrachten, oder als Grenzfall (indem man den Schnittpunkt
von $AA^{\prime}$, $BB^{\prime}$ und $CC^{\prime}$ immer weiter vom Dreieck
$ABC$ wegschiebt). Die Umkehrung erh\"{a}lt man durch Eindeutigkeit (das
Verh\"{a}ltnis $\dfrac{BA^{\prime}}{A^{\prime}C}$ bestimmt die Lage des
Punktes $A^{\prime}$ auf der Geraden $BC$ eindeutig). Details gibt es in den
meisten Geometrieb\"{u}chern.

Anwendungen von Ceva gibt es zuhauf (ich werde evtl. sp\"{a}ter was dazu
posten). Hier sind ein paar:

\begin{exercise}
[Paul Yiu]\label{exe.yiu-inner}Sei $ABC$ ein nicht-entartetes Dreieck. Sei $P$
ein Punkt. Die Innenwinkelhalbierenden der Winkel $BPC$, $CPA$ und $APB$
schneiden die Strecken $BC$, $CA$ bzw. $AB$ jeweils in $X$, $Y$ bzw. $Z$. Man
beweise: Die Geraden $AX$, $BY$ und $CZ$ schneiden sich in einem Punkt. (Siehe
Fig. \ref{fig.2}.)
\end{exercise}

%

%TCIMACRO{\FRAME{ftbpFU}{4.3019in}{4.2968in}{0pt}{\Qcb{Paul Yius
%Innenwinkelhalbierenden}}{\Qlb{fig.2}}{Figure}%
%{\special{ language "Scientific Word";  type "GRAPHIC";
%maintain-aspect-ratio TRUE;  display "USEDEF";  valid_file "T";
%width 4.3019in;  height 4.2968in;  depth 0pt;  original-width 3.9283in;
%original-height 3.9232in;  cropleft "0";  croptop "1";  cropright "1";
%cropbottom "0";  tempfilename 'Yiu1.wmf';tempfile-properties "XNPR";}}}%
%BeginExpansion
\begin{figure}[ptb]%
\centering
\includegraphics[
natheight=3.923200in,
natwidth=3.928300in,
height=4.2968in,
width=4.3019in
]%
{Yiu1.wmf}%
\caption{Paul Yius Innenwinkelhalbierenden}%
\label{fig.2}%
\end{figure}
%EndExpansion


\textit{L\"{o}sung (Florian):} Wir erinnern uns an den Satz, dass eine
Winkelhalbierende im Dreieck die Gegenseite im Verh\"{a}ltnis der anliegenden
Seiten teilt\footnote{Ausf\"{u}hrlich formuliert behauptet dieser Satz
folgendes: Wenn die Innenwinkelhalbierende des Winkels $CAB$ eines Dreiecks
$ABC$ die Seite $BC$ im Punkt $D$ schneidet, dann gilt $\dfrac{BD}{DC}%
=\dfrac{BA}{AC}$.}. Also ist%
\[
\dfrac{BX}{XC}=\dfrac{BP}{CP},\ \ \ \ \ \ \ \ \ \ \dfrac{CY}{YA}=\dfrac
{CP}{AP},\ \ \ \ \ \ \ \ \ \ \text{und}\ \ \ \ \ \ \ \ \ \ \dfrac{AZ}%
{ZB}=\dfrac{AP}{BP}.
\]
Multiplikation dieser drei Gleichungen ergibt%
\[
\dfrac{BX}{XC}\cdot\dfrac{CY}{YA}\cdot\dfrac{AZ}{ZB}=\dfrac{BP}{CP}\cdot
\dfrac{CP}{AP}\cdot\dfrac{AP}{BP}=1.
\]
Nach Ceva schneiden sich also $AX$, $BY$ und $CZ$ in einem Punkt.

\begin{exercise}
[isotomische Punkte]\label{exe.isotomic}Sei $ABC$ ein nicht-entartetes
Dreieck. Sei $P$ ein Punkt. Seien $A^{\prime}$, $B^{\prime}$ und $C^{\prime}$
die Schnittpunkte der Geraden $AP$, $BP$ bzw. $CP$ mit den Geraden $BC$, $CA$
bzw. $AB$. Seien $A^{\prime\prime}$, $B^{\prime\prime}$ und $C^{\prime\prime}$
die Spiegelbilder der Punkte $A^{\prime}$, $B^{\prime}$ bzw. $C^{\prime}$ an
den Mittelpunkten der Strecken $BC$, $CA$ bzw. $AB$. Man zeige: Die Geraden
$AA^{\prime\prime}$, $BB^{\prime\prime}$ und $CC^{\prime\prime}$ schneiden
sich in einem Punkt.
\end{exercise}

Letzterer Punkt hei\ss t der zu $P$ \textit{isotomische} (oder
\textit{isotomisch konjugierte}) Punkt bez\"{u}glich des Dreiecks $ABC$.
(Siehe Fig. \ref{fig.3}.)%

%TCIMACRO{\FRAME{ftbpFU}{5.083in}{4.9771in}{0pt}{\Qcb{Isotomische Punkte}%
%}{\Qlb{fig.3}}{Figure}{\special{ language "Scientific Word";  type "GRAPHIC";
%maintain-aspect-ratio TRUE;  display "USEDEF";  valid_file "T";
%width 5.083in;  height 4.9771in;  depth 0pt;  original-width 4.2892in;
%original-height 4.2003in;  cropleft "0";  croptop "1";  cropright "1";
%cropbottom "0";  tempfilename 'Isotomic1.wmf';tempfile-properties "XNPR";}}}%
%BeginExpansion
\begin{figure}[ptb]%
\centering
\includegraphics[
natheight=4.200300in,
natwidth=4.289200in,
height=4.9771in,
width=5.083in
]%
{Isotomic1.wmf}%
\caption{Isotomische Punkte}%
\label{fig.3}%
\end{figure}
%EndExpansion


\textit{L\"{o}sung:} Nach Definition von $A^{\prime\prime}$ ist $BA^{\prime
\prime}=A^{\prime}C$ und $A^{\prime\prime}C=BA^{\prime}$, also
\[
\dfrac{BA^{\prime\prime}}{A^{\prime\prime}C}=\dfrac{A^{\prime}C}{BA^{\prime}%
}.
\]
Analog kann man erhalten:%
\begin{align*}
\dfrac{CB^{\prime\prime}}{B^{\prime\prime}A}  &  =\dfrac{B^{\prime}%
A}{CB^{\prime}};\\
\dfrac{AC^{\prime\prime}}{C^{\prime\prime}B}  &  =\dfrac{C^{\prime}%
B}{AC^{\prime}}.
\end{align*}
Daher ist%
\begin{align*}
\dfrac{BA^{\prime\prime}}{A^{\prime\prime}C}\cdot\dfrac{CB^{\prime\prime}%
}{B^{\prime\prime}A}\cdot\dfrac{AC^{\prime\prime}}{C^{\prime\prime}B}  &
=\dfrac{A^{\prime}C}{BA^{\prime}}\cdot\dfrac{B^{\prime}A}{CB^{\prime}}%
\cdot\dfrac{C^{\prime}B}{AC^{\prime}}\\
&  =1\diagup\underbrace{\left(  \dfrac{BA^{\prime}}{A^{\prime}C}\cdot
\dfrac{CB^{\prime}}{B^{\prime}A}\cdot\dfrac{AC^{\prime}}{C^{\prime}B}\right)
}_{=1\text{ (nach Ceva)}}\\
&  =1\diagup1=1.
\end{align*}
Nach Ceva folgt die Behauptung.

\begin{exercise}
[Satz von Reuschle-Terquem; isozyklische Punkte]\label{exe.isocyclic}Sei $ABC$
ein nicht-entartetes Dreieck. Sei $P$ ein Punkt. Seien $A^{\prime}$,
$B^{\prime}$ und $C^{\prime}$ die Schnittpunkte der Geraden $AP$, $BP$ bzw.
$CP$ mit den Geraden $BC$, $CA$ bzw. $AB$. Seien $A^{\prime\prime}$,
$B^{\prime\prime}$ und $C^{\prime\prime}$ die zweiten Schnittpunkte des
Umkreises des Dreiecks $A^{\prime}B^{\prime}C^{\prime}$ mit den Geraden $BC$,
$CA$ bzw. $AB$. (Mit \textquotedblleft zweite Schnittpunkte\textquotedblright%
\ meine ich die jeweils von $A^{\prime}$, $B^{\prime}$ bzw. $C^{\prime}$
verschiedenen Schnittpunkte. Wenn der Kreis die Gerade $BC$ ber\"{u}hrt, sind
allerdings die zwei Schnittpunkte als gleich zu verstehen.)

Man zeige: Die Geraden $AA^{\prime\prime}$, $BB^{\prime\prime}$ und
$CC^{\prime\prime}$ schneiden sich in einem Punkt.
\end{exercise}

Letzterer Punkt hei\ss t der zu $P$ \textit{isozyklisch konjugierte} (oder
kurz \textit{isozyklische}) Punkt bez\"{u}glich des Dreiecks $ABC$. Im
Englischen hei\ss t er \textquotedblleft cyclocevian conjugate of
$P$\textquotedblright. (Siehe Fig. \ref{fig.4}.)%

%TCIMACRO{\FRAME{ftbpFU}{5.0906in}{4.5594in}{0pt}{\Qcb{Isoyzklische Punkte}%
%}{\Qlb{fig.4}}{Figure}{\special{ language "Scientific Word";  type "GRAPHIC";
%maintain-aspect-ratio TRUE;  display "USEDEF";  valid_file "T";
%width 5.0906in;  height 4.5594in;  depth 0pt;  original-width 4.5806in;
%original-height 4.1003in;  cropleft "0";  croptop "1";  cropright "1";
%cropbottom "0";  tempfilename 'Cyclocevian1.wmf';tempfile-properties "XNPR";}%
%}}%
%BeginExpansion
\begin{figure}[ptb]%
\centering
\includegraphics[
natheight=4.100300in,
natwidth=4.580600in,
height=4.5594in,
width=5.0906in
]%
{Cyclocevian1.wmf}%
\caption{Isoyzklische Punkte}%
\label{fig.4}%
\end{figure}
%EndExpansion


\textit{L\"{o}sung (Richard):} Die Punkte $B^{\prime}$, $B^{\prime\prime}$,
$C^{\prime}$ und $C^{\prime\prime}$ liegen auf einem Kreis. Nach dem
Sekantensatz gilt also%
\[
AC^{\prime}\cdot AC^{\prime\prime}=AB^{\prime}\cdot AB^{\prime\prime
}=B^{\prime}A\cdot B^{\prime\prime}A,
\]
wobei die Strecken gerichtet sind. Analog gilt%
\begin{align*}
BA^{\prime}\cdot BA^{\prime\prime}  &  =C^{\prime}B\cdot C^{\prime\prime}B;\\
CB^{\prime}\cdot CB^{\prime\prime}  &  =A^{\prime}C\cdot A^{\prime\prime}C.
\end{align*}
Hieraus folgt schnell%
\begin{align*}
&  \left(  \dfrac{BA^{\prime}}{A^{\prime}C}\cdot\dfrac{CB^{\prime}}{B^{\prime
}A}\cdot\dfrac{AC^{\prime}}{C^{\prime}B}\right)  \cdot\left(  \dfrac
{BA^{\prime\prime}}{A^{\prime\prime}C}\cdot\dfrac{CB^{\prime\prime}}%
{B^{\prime\prime}A}\cdot\dfrac{AC^{\prime\prime}}{C^{\prime\prime}B}\right) \\
&  =\underbrace{\dfrac{AC^{\prime}\cdot AC^{\prime\prime}}{B^{\prime}A\cdot
B^{\prime\prime}A}}_{=1}\cdot\underbrace{\dfrac{BA^{\prime}\cdot
BA^{\prime\prime}}{C^{\prime}B\cdot C^{\prime\prime}B}}_{=1}\cdot
\underbrace{\dfrac{CB^{\prime}\cdot CB^{\prime\prime}}{A^{\prime}C\cdot
A^{\prime\prime}C}}_{=1}\\
&  =1.
\end{align*}
Da $\dfrac{BA^{\prime}}{A^{\prime}C}\cdot\dfrac{CB^{\prime}}{B^{\prime}A}%
\cdot\dfrac{AC^{\prime}}{C^{\prime}B}=1$ ist (nach Ceva), folgt hieraus, dass
auch $\dfrac{BA^{\prime\prime}}{A^{\prime\prime}C}\cdot\dfrac{CB^{\prime
\prime}}{B^{\prime\prime}A}\cdot\dfrac{AC^{\prime\prime}}{C^{\prime\prime}%
B}=1$ ist. Nach Ceva folgt die Behauptung.

\subsection{Menelaos}

Ein Gegenst\"{u}ck zum Satz von Ceva ist der Satz von Menelaos:

\begin{theorem}
[Satz von Menelaos]\label{thm.menelaos}Sei $ABC$ ein nicht-entartetes Dreieck.
Seien $A^{\prime}$, $B^{\prime}$ und $C^{\prime}$ drei Punkte jeweils auf den
Geraden $BC$, $CA$ bzw. $AB$. Genau dann liegen die Punkte $A^{\prime}$,
$B^{\prime}$ und $C^{\prime}$ auf einer Geraden, wenn%
\[
\dfrac{BA^{\prime}}{A^{\prime}C}\cdot\dfrac{CB^{\prime}}{B^{\prime}A}%
\cdot\dfrac{AC^{\prime}}{C^{\prime}B}=-1
\]
gilt. (Siehe Fig. \ref{fig.5}.)
\end{theorem}

%

%TCIMACRO{\FRAME{ftbpFU}{5.172in}{3.5259in}{0pt}{\Qcb[Satz von Menelaos]{Satz
%von Menelaos}}{\Qlb{fig.5}}{Figure}{\special{ language "Scientific Word";
%type "GRAPHIC";  maintain-aspect-ratio TRUE;  display "USEDEF";
%valid_file "T";  width 5.172in;  height 3.5259in;  depth 0pt;
%original-width 4.6188in;  original-height 3.1405in;  cropleft "0";
%croptop "1";  cropright "1";  cropbottom "0";
%tempfilename 'Menelaos1.wmf';tempfile-properties "XNPR";}}}%
%BeginExpansion
\begin{figure}[ptb]%
\centering
\includegraphics[
natheight=3.140500in,
natwidth=4.618800in,
height=3.5259in,
width=5.172in
]%
{Menelaos1.wmf}%
\caption[Satz von Menelaos]{Satz von Menelaos}%
\label{fig.5}%
\end{figure}
%EndExpansion


\begin{exercise}
[Paul Yiu]\label{exe.yiu-outer}Sei $ABC$ ein nicht-entartetes Dreieck. Sei $P$
ein Punkt. Die \textbf{Au\ss en}winkelhalbierenden der Winkel $BPC$, $CPA$ und
$APB$ schneiden die Geraden $BC$, $CA$ bzw. $AB$ jeweils in $X$, $Y$ bzw. $Z$.
Man zeige: Die Punkte $X$, $Y$ und $Z$ liegen auf einer Geraden.
\end{exercise}

\textit{L\"{o}sung:} Eine \"{a}hnliche Aufgabe haben wir bereits f\"{u}r die
Innenwinkelhalbierenden gel\"{o}st (Aufgabe \ref{exe.yiu-inner}), aber jetzt
m\"{u}ssen wir nat\"{u}rlich Menelaos statt Ceva anwenden.

F\"{u}r die Au\ss enwinkelhalbierende gilt das gleiche Abstandsverh\"{a}ltnis
wie f\"{u}r die Innenwinkelhalbierende, nur mit einem Minuszeichen. Also gilt
$\dfrac{BX}{XC}=-\dfrac{BP}{PC}$ und so weiter. Folglich ist%
\[
\dfrac{BX}{XC}\cdot\dfrac{CY}{YA}\cdot\dfrac{AZ}{ZB}=\left(  -\dfrac{BP}%
{CP}\right)  \cdot\left(  -\dfrac{CP}{AP}\right)  \cdot\left(  -\dfrac{AP}%
{BP}\right)  =-1,
\]
und man kann Menelaos anwenden.

\begin{exercise}
\label{exe.isotomic-line}\textbf{(a)} Zeige ein Analogon zum Satz vom
isotomischen Punkt (Aufgabe \ref{exe.isotomic}), bei dem Menelaos statt Ceva
verwendet wird.

\textbf{(b)} Warum gibt es kein solches zum Satz vom isozyklischen Punkt?
\end{exercise}

\textit{L\"{o}sung (Christian):} \textbf{(a)} Hier ist die analoge Aufgabe:
Eine Gerade schneide die Seitengeraden $BC$, $CA$ und $AB$ eines Dreiecks
$ABC$ in den Punkten $A^{\prime}$, $B^{\prime}$ bzw. $C^{\prime}$. Seien
$A^{\prime\prime}$, $B^{\prime\prime}$ und $C^{\prime\prime}$ die
Spiegelbilder der Punkte $A^{\prime}$, $B^{\prime}$ bzw. $C^{\prime}$ an den
Mittelpunkten der Strecken $BC$, $CA$ bzw. $AB$. Man zeige: Die Punkte
$A^{\prime\prime}$, $B^{\prime\prime}$ und $C^{\prime\prime}$ liegen auf einer Geraden.

Der Beweis hiervon ist fast derselbe wie f\"{u}r Aufgabe \ref{exe.isotomic},
verwendet aber Menelaos statt Ceva.

\textbf{(b)} Der Umkreis eines entarteten Dreiecks $A^{\prime}B^{\prime
}C^{\prime}$ ist undefiniert oder eine Gerade; damit bekommt man hier kein
sinnvolles Resultat.

Im Folgenden bezeichnet $g\cap h$ den Schnittpunkt zweier Geraden $g$ und $h$.

\begin{exercise}
[Satz von Pascal]\label{exe.pascal}Seien $A$, $B$, $C$, $D$, $E$ und $F$ sechs
Punkte auf einem Kreis. Man zeige: Die Schnittpunkte%
\[
AB\cap DE,\ \ \ \ \ \ \ \ \ \ BC\cap EF\ \ \ \ \ \ \ \ \ \ \text{und}%
\ \ \ \ \ \ \ \ \ \ CD\cap FA
\]
liegen auf einer Geraden. (Siehe Fig. \ref{fig.6}.)
\end{exercise}

%

%TCIMACRO{\FRAME{ftbpFU}{7.1713in}{7.5296in}{0pt}{\Qcb{Satz von Pascal}%
%}{\Qlb{fig.6}}{Figure}{\special{ language "Scientific Word";  type "GRAPHIC";
%maintain-aspect-ratio TRUE;  display "USEDEF";  valid_file "T";
%width 7.1713in;  height 7.5296in;  depth 0pt;  original-width 6.33in;
%original-height 6.6486in;  cropleft "0";  croptop "1";  cropright "1";
%cropbottom "0";  tempfilename 'Pascal2.wmf';tempfile-properties "XNPR";}}}%
%BeginExpansion
\begin{figure}[ptb]%
\centering
\includegraphics[
natheight=6.648600in,
natwidth=6.330000in,
height=7.5296in,
width=7.1713in
]%
{Pascal2.wmf}%
\caption{Satz von Pascal}%
\label{fig.6}%
\end{figure}
%EndExpansion


\textit{L\"{o}sung:} Seien $X=AB\cap DE$, $Y=BC\cap EF$ und $Z=CD\cap FA$.
Seien ferner $P=FA\cap BC$ und $Q=BC\cap DE$ und $R=DE\cap FA$. Wir wollen nun
Menelaos im Dreieck $PQR$ anwenden (auf dessen Seiten $QR$, $RP$ und $PQ$
jeweils die Punkte $X$, $Z$ bzw. $Y$ liegen). Wir m\"{u}ssen dazu zeigen:%
\[
\dfrac{QX}{XR}\cdot\dfrac{RZ}{ZP}\cdot\dfrac{PY}{YQ}=-1.
\]


Doch Menelaos, angewandt auf das Dreieck $PQR$ und die Gerade $CZD$, ergibt%
\begin{align*}
\dfrac{RZ}{ZP}\cdot\dfrac{PC}{CQ}\cdot\dfrac{QD}{DR}  &
=-1,\ \ \ \ \ \ \ \ \ \ \text{also}\\
\dfrac{RZ}{ZP}  &  =-1\diagup\left(  \dfrac{PC}{CQ}\cdot\dfrac{QD}{DR}\right)
=-\dfrac{DR}{QD}\cdot\dfrac{CQ}{PC}.
\end{align*}
Analog ist%
\begin{align*}
\dfrac{PY}{YQ}  &  =-\dfrac{FP}{RF}\cdot\dfrac{ER}{QE}%
\ \ \ \ \ \ \ \ \ \ \text{und}\\
\dfrac{QX}{XR}  &  =-\dfrac{BQ}{PB}\cdot\dfrac{AP}{RA}.
\end{align*}
Multiplikation dieser drei Gleichungen ergibt%
\begin{align*}
&  \dfrac{QX}{XR}\cdot\dfrac{RZ}{ZP}\cdot\dfrac{PY}{YQ}\\
&  =\left(  -\dfrac{DR}{QD}\cdot\dfrac{CQ}{PC}\right)  \cdot\left(
-\dfrac{FP}{RF}\cdot\dfrac{ER}{QE}\right)  \cdot\left(  -\dfrac{BQ}{PB}%
\cdot\dfrac{AP}{RA}\right) \\
&  =-\dfrac{AP\cdot FP}{PC\cdot PB}\cdot\dfrac{DR\cdot ER}{RA\cdot RF}%
\cdot\dfrac{CQ\cdot BQ}{QE\cdot QD}=-1,
\end{align*}
weil nach dem Sekantensatz jeder der Br\"{u}che gleich $1$ ist. Nach Menelaos
folgt hieraus die Behauptung.

\begin{exercise}
[Satz von Gauss]\label{exe.gauss}Sei $ABC$ ein Dreieck. Seien $A^{\prime}$,
$B^{\prime}$ und $C^{\prime}$ drei Punkte jeweils auf den Geraden $BC$, $CA$
bzw. $AB$, die auf einer Geraden liegen. Man zeige: Die Mittelpunkte der
Strecken $AA^{\prime}$, $BB^{\prime}$ und $CC^{\prime}$ liegen ebenfalls auf
einer Geraden. (Siehe Fig. \ref{fig.7}.)
\end{exercise}

%

%TCIMACRO{\FRAME{ftbpFU}{6.0107in}{5.111in}{0pt}{\Qcb{Satz von Gauss}%
%}{\Qlb{fig.7}}{Figure}{\special{ language "Scientific Word";  type "GRAPHIC";
%maintain-aspect-ratio TRUE;  display "USEDEF";  valid_file "T";
%width 6.0107in;  height 5.111in;  depth 0pt;  original-width 5.3948in;
%original-height 4.5832in;  cropleft "0";  croptop "1";  cropright "1";
%cropbottom "0";  tempfilename 'Gauss1.wmf';tempfile-properties "XNPR";}}}%
%BeginExpansion
\begin{figure}[ptb]%
\centering
\includegraphics[
natheight=4.583200in,
natwidth=5.394800in,
height=5.111in,
width=6.0107in
]%
{Gauss1.wmf}%
\caption{Satz von Gauss}%
\label{fig.7}%
\end{figure}
%EndExpansion


\textit{L\"{o}sung:} (Siehe Fig. \ref{fig.8}.) Seien $M_{a}$, $M_{b}$ und
$M_{c}$ die Mittelpunkte der Seiten $BC$, $CA$ bzw. $AB$. Dann ist
$XM_{c}\parallel BA^{\prime}$ (als Mittelparallele im Dreieck $ABA^{\prime}$)
und $XM_{b}\parallel CA^{\prime}$ (analog). Dies bedeutet, dass beide Geraden
$XM_{c}$ und $XM_{b}$ zu $BC$ parallel sind. Also fallen diese beiden Geraden
zusammen. Der Punkt $X$ liegt daher auf der Geraden $M_{b}M_{c}$. Analog
liegen die Punkte $Y$ und $Z$ auf den Geraden $M_{c}M_{a}$ bzw. $M_{a}M_{b}$.
Ferner ist $M_{b}M_{c}\parallel BC$ (denn $XM_{c}\parallel BA^{\prime}$), und
nach dem Strahlensatz ist somit%
\[
\dfrac{M_{b}X}{XM_{c}}=\dfrac{CA^{\prime}}{A^{\prime}B}.
\]
Analog finden wir%
\[
\dfrac{M_{c}Y}{YM_{a}}=\dfrac{AB^{\prime}}{B^{\prime}C}%
\ \ \ \ \ \ \ \ \ \ \text{und}\ \ \ \ \ \ \ \ \ \ \dfrac{M_{a}Z}{ZM_{b}%
}=\dfrac{BC^{\prime}}{C^{\prime}A}.
\]
Multiplikation dieser drei Gleichungen ergibt%
\[
\dfrac{M_{b}X}{XM_{c}}\cdot\dfrac{M_{c}Y}{YM_{a}}\cdot\dfrac{M_{a}Z}{ZM_{b}%
}=\dfrac{CA^{\prime}}{A^{\prime}B}\cdot\dfrac{AB^{\prime}}{B^{\prime}C}%
\cdot\dfrac{BC^{\prime}}{C^{\prime}A}=-1
\]
nach Menelaos (weil $A^{\prime}$, $B^{\prime}$ und $C^{\prime}$ auf einer
Geraden liegen). Nach Menelaos (f\"{u}r Dreieck $M_{a}M_{b}M_{c}$) liegen also
$X$, $Y$ und $Z$ auf einer Geraden (denn $X$, $Y$ und $Z$ liegen auf den
Geraden $M_{b}M_{c}$, $M_{c}M_{a}$ bzw. $M_{a}M_{b}$). Damit ist die Aufgabe gel\"{o}st.%

%TCIMACRO{\FRAME{ftbpFU}{5.9048in}{4.7136in}{0pt}{\Qcb{Beweis des Satzes von
%Gauss}}{\Qlb{fig.8}}{Figure}{\special{ language "Scientific Word";
%type "GRAPHIC";  maintain-aspect-ratio TRUE;  display "USEDEF";
%valid_file "T";  width 5.9048in;  height 4.7136in;  depth 0pt;
%original-width 5.0949in;  original-height 4.0613in;  cropleft "0";
%croptop "1";  cropright "1";  cropbottom "0";
%tempfilename 'Gauss2.wmf';tempfile-properties "XNPR";}}}%
%BeginExpansion
\begin{figure}[ptb]%
\centering
\includegraphics[
natheight=4.061300in,
natwidth=5.094900in,
height=4.7136in,
width=5.9048in
]%
{Gauss2.wmf}%
\caption{Beweis des Satzes von Gauss}%
\label{fig.8}%
\end{figure}
%EndExpansion


\begin{exercise}
[Satz von Desargues]\label{exe.desargues}Seien $ABC$ und $A^{\prime}B^{\prime
}C^{\prime}$ zwei nicht-entartete Dreiecke. Man zeige: Die Geraden
$AA^{\prime}$, $BB^{\prime}$ und $CC^{\prime}$ schneiden sich genau dann in
einem Punkt, wenn die Punkte%
\[
BC\cap B^{\prime}C^{\prime},\ \ \ \ \ \ \ \ \ \ CA\cap C^{\prime}A^{\prime
}\ \ \ \ \ \ \ \ \ \ \text{und}\ \ \ \ \ \ \ \ \ \ AB\cap A^{\prime}B^{\prime}%
\]
auf einer Geraden liegen. (Hierbei wird angenommen, dass die Situation nicht
\textquotedblleft zu stark ausgeartet ist\textquotedblright\ -- so d\"{u}rfen
die Punkte $A$ und $A^{\prime}$ beispielsweise nicht zusammenfallen.) (Siehe
Fig. \ref{fig.9}.)
\end{exercise}

%

%TCIMACRO{\FRAME{ftbpFU}{4.795in}{6.2258in}{0pt}{\Qcb{Satz von Desargues}%
%}{\Qlb{fig.9}}{Figure}{\special{ language "Scientific Word";  type "GRAPHIC";
%maintain-aspect-ratio TRUE;  display "USEDEF";  valid_file "T";
%width 4.795in;  height 6.2258in;  depth 0pt;  original-width 4.3197in;
%original-height 5.6201in;  cropleft "0";  croptop "1";  cropright "1";
%cropbottom "0";  tempfilename 'Desargues1.wmf';tempfile-properties "XNPR";}}}%
%BeginExpansion
\begin{figure}[ptb]%
\centering
\includegraphics[
natheight=5.620100in,
natwidth=4.319700in,
height=6.2258in,
width=4.795in
]%
{Desargues1.wmf}%
\caption{Satz von Desargues}%
\label{fig.9}%
\end{figure}
%EndExpansion


\textit{L\"{o}sung (Boldizsar):} Seien%
\[
X=BC\cap B^{\prime}C^{\prime},\ \ \ \ \ \ \ \ \ \ Y=CA\cap C^{\prime}%
A^{\prime}\ \ \ \ \ \ \ \ \ \ \text{und}\ \ \ \ \ \ \ \ \ \ Z=AB\cap
A^{\prime}B^{\prime}.
\]
Wir m\"{u}ssen also folgendes zeigen: Die Geraden $AA^{\prime}$, $BB^{\prime}$
und $CC^{\prime}$ schneiden sich genau dann in einem Punkt, wenn die Punkte
$X$, $Y$ und $Z$ auf einer Geraden liegen.

Von dieser Behauptung beweisen wir die \textquotedblleft$\Longrightarrow
$\textquotedblright-Implikation und die \textquotedblleft$\Longleftarrow
$\textquotedblright-Implikation separat:

$\Longrightarrow:$ Angenommen, die Geraden $AA^{\prime}$, $BB^{\prime}$ und
$CC^{\prime}$ schneiden sich in einem Punkt. Wir nennen diesen Punkt $P$.

Menelaos im Dreieck $BPC$ f\"{u}r die Gerade $B^{\prime}C^{\prime}X$ ergibt%
\[
\dfrac{BX}{XC}\cdot\dfrac{CC^{\prime}}{C^{\prime}P}\cdot\dfrac{PB^{\prime}%
}{B^{\prime}B}=-1,
\]
also%
\[
\dfrac{BX}{XC}=-\dfrac{B^{\prime}B}{PB^{\prime}}\cdot\dfrac{C^{\prime}%
P}{CC^{\prime}}=-\dfrac{BB^{\prime}}{B^{\prime}P}\diagup\dfrac{CC^{\prime}%
}{C^{\prime}P}.
\]
Durch zyklisches Vertauschen der Ecken $A$, $B$ und $C$ (und dementsprechend
auch der Ecken $A^{\prime}$, $B^{\prime}$ und $C^{\prime}$ sowie der Punkte
$X$, $Y$ bzw. $Z$) erh\"{a}lt man aus dieser Gleichung die zwei analogen
Gleichungen%
\begin{align*}
\dfrac{CY}{YA}  &  =-\dfrac{CC^{\prime}}{C^{\prime}P}\diagup\dfrac{AA^{\prime
}}{A^{\prime}P}\ \ \ \ \ \ \ \ \ \ \text{und}\\
\dfrac{AZ}{ZB}  &  =-\dfrac{AA^{\prime}}{A^{\prime}P}\diagup\dfrac{BB^{\prime
}}{B^{\prime}P}.
\end{align*}
Multiplikation all dieser drei Gleichungen ergibt%
\begin{align*}
&  \dfrac{BX}{XC}\cdot\dfrac{CY}{YA}\cdot\dfrac{AZ}{ZB}\\
&  =\left(  -\dfrac{BB^{\prime}}{B^{\prime}P}\diagup\dfrac{CC^{\prime}%
}{C^{\prime}P}\right)  \cdot\left(  -\dfrac{CC^{\prime}}{C^{\prime}P}%
\diagup\dfrac{AA^{\prime}}{A^{\prime}P}\right)  \cdot\left(  -\dfrac
{AA^{\prime}}{A^{\prime}P}\diagup\dfrac{BB^{\prime}}{B^{\prime}P}\right)  =-1
\end{align*}
(Teleskopprodukt). Nach Menelaos folgt hieraus, dass $X$, $Y$ und $Z$ auf
einer Geraden liegen. Damit ist die \textquotedblleft$\Longrightarrow
$\textquotedblright-Richtung bewiesen.

$\Longleftarrow:$ Hier gibt es verschiedene M\"{o}glichkeiten. Die vielleicht
eleganteste ist folgende: Angenommen, die Punkte $X$, $Y$ und $Z$ liegen auf
einer Geraden. Die Geraden $BC$, $B^{\prime}C^{\prime}$ und $ZY$ schneiden
sich somit in einem Punkt (n\"{a}mlich im Punkt $X$). Somit k\"{o}nnen wir die
\textquotedblleft$\Longrightarrow$\textquotedblright-Richtung unserer Aufgabe
auf die Dreiecke $BB^{\prime}Z$ und $CC^{\prime}Y$ anstelle von den Dreiecken
$ABC$ bzw. $A^{\prime}B^{\prime}C^{\prime}$ anwenden. Wir erhalten dadurch,
dass die Punkte%
\[
B^{\prime}Z\cap C^{\prime}Y,\ \ \ \ \ \ \ \ \ \ ZB\cap
YC\ \ \ \ \ \ \ \ \ \ \text{und}\ \ \ \ \ \ \ \ \ \ BB^{\prime}\cap
CC^{\prime}%
\]
auf einer Geraden liegen. Doch diese drei Punkte sind $A^{\prime}$, $A$ und
$BB^{\prime}\cap CC^{\prime}$. Somit liegen die Punkte $A^{\prime}$, $A$ und
$BB^{\prime}\cap CC^{\prime}$ auf einer Geraden. Mit anderen Worten: Der Punkt
$BB^{\prime}\cap CC^{\prime}$ liegt auf der Geraden $AA^{\prime}$. Das
hei\ss t, die Geraden $AA^{\prime}$, $BB^{\prime}$ und $CC^{\prime}$ schneiden
sich in einem Punkt. Damit ist auch die \textquotedblleft$\Longleftarrow
$\textquotedblright-Richtung bewiesen.

\subsection{Trig-Ceva}

Folgende Variante des Satzes von Ceva erlaubt uns, die Konkurrenz von
Ecktransversalen zu zeigen, ohne ihre Schnittpunkte mit den Seiten zu kennen.

\begin{theorem}
[Trigonometrische Version des Satzes von Ceva, oder kurz Satz von
Trig-Ceva]\label{thm.trigceva}Sei $ABC$ ein nicht-entartetes Dreieck, und
seien $X$, $Y$ und $Z$ drei Punkte in seiner Ebene mit $X\neq A$ und $Y\neq B$
und $Z\neq C$. Genau dann schneiden sich die Geraden $AX$, $BY$ und $CZ$ in
einem Punkt, wenn%
\[
\dfrac{\sin\measuredangle BAX}{\sin\measuredangle XAC}\cdot\dfrac
{\sin\measuredangle CBY}{\sin\measuredangle YBA}\cdot\dfrac{\sin\measuredangle
ACZ}{\sin\measuredangle ZCB}=1
\]
gilt. Hierbei sind die Winkel orientiert (und k\"{o}nnen gerne modulo
$360^{\circ}$ verstanden werden). (Siehe Fig. \ref{fig.10}.)
\end{theorem}

%

%TCIMACRO{\FRAME{ftbpFU}{4.3477in}{3.1506in}{0pt}{\Qcb{{}Satz von Trig-Ceva}%
%}{\Qlb{fig.10}}{Figure}{\special{ language "Scientific Word";
%type "GRAPHIC";  maintain-aspect-ratio TRUE;  display "USEDEF";
%valid_file "T";  width 4.3477in;  height 3.1506in;  depth 0pt;
%original-width 4.2053in;  original-height 3.0396in;  cropleft "0";
%croptop "1";  cropright "1";  cropbottom "0";
%tempfilename 'TrigCeva1.wmf';tempfile-properties "XNPR";}}}%
%BeginExpansion
\begin{figure}[ptb]%
\centering
\includegraphics[
natheight=3.039600in,
natwidth=4.205300in,
height=3.1506in,
width=4.3477in
]%
{TrigCeva1.wmf}%
\caption{{}Satz von Trig-Ceva}%
\label{fig.10}%
\end{figure}
%EndExpansion


\textit{Beweis:} Man kann Trig-Ceva aus Ceva herleiten. Aber es geht noch
einfacher: F\"{u}r jeden Punkt $P$ und jede Gerade $g$ bezeichnen wir mit
$d\left(  P;\ g\right)  $ den orientierten Abstand von $P$ zur Geraden $g$.
Dabei hei\ss t \textquotedblleft orientiert\textquotedblright, dass wir zu
jeder Geraden eine Halbebene w\"{a}hlen, in der dieser Abstand positiv sein
soll, w\"{a}hrend er in der gegen\"{u}berliegenden Halbebene negativ sein
soll. Dabei w\"{a}hlen wir die Halbebenen f\"{u}r die Geraden $BC$, $CA$ und
$AB$ so, dass die Punkte $A$, $B$ bzw. $C$ jeweils in den positiven Halbebenen
liegen (d.h., dass die orientierten Abst\"{a}nde $d\left(  A;\ BC\right)  $,
$d\left(  B;\ CA\right)  $ und $d\left(  C;\ AB\right)  $ positiv sind).%

%TCIMACRO{\FRAME{ftbpFU}{4.5925in}{3.2862in}{0pt}{\Qcb{{}Beweis des Satzes von
%Trig-Ceva}}{\Qlb{fig.11}}{Figure}{\special{ language "Scientific Word";
%type "GRAPHIC";  maintain-aspect-ratio TRUE;  display "USEDEF";
%valid_file "T";  width 4.5925in;  height 3.2862in;  depth 0pt;
%original-width 4.0672in;  original-height 2.9024in;  cropleft "0";
%croptop "1";  cropright "1";  cropbottom "0";
%tempfilename 'TrigCeva2.wmf';tempfile-properties "XNPR";}}}%
%BeginExpansion
\begin{figure}[ptb]%
\centering
\includegraphics[
natheight=2.902400in,
natwidth=4.067200in,
height=3.2862in,
width=4.5925in
]%
{TrigCeva2.wmf}%
\caption{{}Beweis des Satzes von Trig-Ceva}%
\label{fig.11}%
\end{figure}
%EndExpansion


Wir sehen nun leicht ein, dass $\dfrac{\sin\measuredangle BAQ}{\sin
\measuredangle QAC}=\dfrac{d\left(  Q;\ CA\right)  }{d\left(  Q;\ AB\right)
}$ f\"{u}r jeden Punkt $Q$ gilt (siehe Fig. \ref{fig.11}). Nehmen wir nun an,
dass die Geraden $AX$, $BY$ und $CZ$ sich in einem Punkt $Q$ schneiden. Dann
ist%
\begin{align*}
\dfrac{\sin\measuredangle BAX}{\sin\measuredangle XAC}  &  =\dfrac
{\sin\measuredangle BAQ}{\sin\measuredangle QAC}=\dfrac{d\left(
Q;\ AB\right)  }{d\left(  Q;\ CA\right)  }\ \ \ \ \ \ \ \ \ \ \text{und}\\
\dfrac{\sin\measuredangle CBY}{\sin\measuredangle YBA}  &  =\dfrac
{\sin\measuredangle CBQ}{\sin\measuredangle QBA}=\dfrac{d\left(
Q;\ BC\right)  }{d\left(  Q;\ AB\right)  }\ \ \ \ \ \ \ \ \ \ \text{und}\\
\dfrac{\sin\measuredangle ACZ}{\sin\measuredangle ZCB}  &  =\dfrac
{\sin\measuredangle ACQ}{\sin\measuredangle QCB}=\dfrac{d\left(
Q;\ CA\right)  }{d\left(  Q;\ BC\right)  }.
\end{align*}
Multiplikation dieser drei Gleichungen ergibt%
\[
\dfrac{\sin\measuredangle BAX}{\sin\measuredangle XAC}\cdot\dfrac
{\sin\measuredangle CBY}{\sin\measuredangle YBA}\cdot\dfrac{\sin\measuredangle
ACZ}{\sin\measuredangle ZCB}=\dfrac{d\left(  Q;\ AB\right)  }{d\left(
Q;\ CA\right)  }\cdot\dfrac{d\left(  Q;\ BC\right)  }{d\left(  Q;\ AB\right)
}\cdot\dfrac{d\left(  Q;\ CA\right)  }{d\left(  Q;\ BC\right)  }=1.
\]
Damit ist eine Richtung von Trig-Ceva bewiesen. Die andere folgt leicht durch
einen Umkehrschluss, bei dem folgendes verwendet wird: Zwei Punkte $Q$ und
$Q^{\prime}$ liegen genau dann auf einer Geraden mit $A$, wenn $\dfrac
{d\left(  Q;\ AB\right)  }{d\left(  Q;\ CA\right)  }=\dfrac{d\left(
Q^{\prime};\ AB\right)  }{d\left(  Q^{\prime};\ CA\right)  }$ gilt (warum?).

\begin{theorem}
\label{thm.jacobi}Sei $ABC$ ein Dreieck. Auf seinen Seiten $BC$, $CA$ und $AB$
werden Dreiecke $BXC$, $CYA$ bzw. $AZB$ derart aufgesetzt, dass%
\begin{align*}
\measuredangle ZAB  &  =\measuredangle CAY\neq0^{\circ},\\
\measuredangle XBC  &  =\measuredangle ABZ\neq0^{\circ}%
\ \ \ \ \ \ \ \ \ \ \text{und}\\
\measuredangle YCA  &  =\measuredangle BCX\neq0^{\circ}%
\end{align*}
(mit orientierten Winkeln) gilt. Dann schneiden sich die Geraden $AX$, $BY$
und $CZ$ in einem Punkt. (Siehe Fig. \ref{fig.12}.)
\end{theorem}

Man beachte, dass die \textquotedblleft$\neq0^{\circ}$\textquotedblright%
-Bedingungen in Satz \ref{thm.jacobi} wichtig sind. W\"{a}ren zum Beispiel die
sechs Winkel $\measuredangle ZAB$, $\measuredangle CAY$, $\measuredangle XBC$
usw. alle gleich $0^{\circ}$, dann w\"{a}ren $X$, $Y$ und $Z$ einfach
irgendwelche Punkte auf den Geraden $BC$, $CA$ bzw. $AB$ (ihre genauere Lage
auf diesen Geraden w\"{a}re durch die Winkel nicht bestimmt), und die
Behauptung von Satz \ref{thm.jacobi} w\"{u}rde meistens nicht gelten.%

%TCIMACRO{\FRAME{ftbpFU}{5.1042in}{5.2923in}{0pt}{\Qcb{Satz von Jacobi}%
%}{\Qlb{fig.12}}{Figure}{\special{ language "Scientific Word";
%type "GRAPHIC";  maintain-aspect-ratio TRUE;  display "USEDEF";
%valid_file "T";  width 5.1042in;  height 5.2923in;  depth 0pt;
%original-width 4.7958in;  original-height 4.9754in;  cropleft "0";
%croptop "1";  cropright "1";  cropbottom "0";
%tempfilename 'Jacobi1.wmf';tempfile-properties "XNPR";}}}%
%BeginExpansion
\begin{figure}[ptb]%
\centering
\includegraphics[
natheight=4.975400in,
natwidth=4.795800in,
height=5.2923in,
width=5.1042in
]%
{Jacobi1.wmf}%
\caption{Satz von Jacobi}%
\label{fig.12}%
\end{figure}
%EndExpansion


\textit{Beweis von Satz \ref{thm.jacobi}:} Seien $\alpha=\measuredangle CAB$,
$\beta=\measuredangle ABC$ und $\gamma=\measuredangle BCA$. Seien%
\[
x=\measuredangle ZAB=\measuredangle CAY,\ \ \ \ \ \ \ \ \ \ y=\measuredangle
XBC=\measuredangle ABZ\ \ \ \ \ \ \ \ \ \ \text{und}%
\ \ \ \ \ \ \ \ \ \ z=\measuredangle YCA=\measuredangle BCX.
\]


Aus Trig-Ceva folgt%
\[
\dfrac{\sin\measuredangle BAX}{\sin\measuredangle XAC}\cdot\dfrac
{\sin\measuredangle CBX}{\sin\measuredangle XBA}\cdot\dfrac{\sin\measuredangle
ACX}{\sin\measuredangle XCB}=1,
\]
da sich die Geraden $AX$, $BX$ und $CX$ in einem Punkt schneiden. Aufl\"{o}sen
nach $\dfrac{\sin\measuredangle BAX}{\sin\measuredangle XAC}$ ergibt%
\begin{align*}
\dfrac{\sin\measuredangle BAX}{\sin\measuredangle XAC}  &  =1\diagup\left(
\dfrac{\sin\measuredangle CBX}{\sin\measuredangle XBA}\cdot\dfrac
{\sin\measuredangle ACX}{\sin\measuredangle XCB}\right) \\
&  =1\diagup\left(  \dfrac{-\sin y}{\sin\left(  y+\beta\right)  }\cdot
\dfrac{\sin\left(  z+\gamma\right)  }{-\sin z}\right) \\
&  =\dfrac{\sin\left(  y+\beta\right)  }{\sin y}\diagup\dfrac{\sin\left(
z+\gamma\right)  }{\sin z}.
\end{align*}
Analoge Gleichungen lassen sich f\"{u}r $\dfrac{\sin\measuredangle CBY}%
{\sin\measuredangle YBA}$ und $\dfrac{\sin\measuredangle ACZ}{\sin
\measuredangle ZCB}$ aufstellen. Multiplikation dieser drei Gleichungen
ergibt
\begin{align*}
&  \dfrac{\sin\measuredangle BAX}{\sin\measuredangle XAC}\cdot\dfrac
{\sin\measuredangle CBX}{\sin\measuredangle XBA}\cdot\dfrac{\sin\measuredangle
ACX}{\sin\measuredangle XCB}\\
&  =\left(  \dfrac{\sin\left(  y+\beta\right)  }{\sin y}\diagup\dfrac
{\sin\left(  z+\gamma\right)  }{\sin z}\right)  \cdot\left(  \dfrac
{\sin\left(  z+\gamma\right)  }{\sin z}\diagup\dfrac{\sin\left(
x+\alpha\right)  }{\sin x}\right) \\
&  \ \ \ \ \ \ \ \ \ \ \cdot\left(  \dfrac{\sin\left(  x+\alpha\right)  }{\sin
x}\diagup\dfrac{\sin\left(  y+\beta\right)  }{\sin y}\right) \\
&  =1\ \ \ \ \ \ \ \ \ \ \left(  \text{weil Teleskopprodukt}\right)  .
\end{align*}
Aus Trig-Ceva folgt somit, dass die Geraden $AX$, $BY$ und $CZ$ sich in einem
Punkt schneiden.

Aus Trig-Ceva folgt leicht noch eine andere Version von Ceva:

\begin{theorem}
[Satz von Ceva auf dem Kreis]\label{thm.kreisceva}Seien $A$, $B$, $C$, $X$,
$Y$ und $Z$ sechs paarweise verschiedene Punkte auf einem Kreis. Genau dann
schneiden sich die Geraden $AX$, $BY$ und $CZ$ in einem Punkt, wenn%
\[
\dfrac{BX}{XC}\cdot\dfrac{CY}{YA}\cdot\dfrac{AZ}{ZB}=1
\]
gilt. Hierbei sind die Strecken orientiert, und zwar folgenderma\ss en: Die
Strecken $BX$, $XC$, $CY$, $YA$, $AZ$ und $ZB$ sollen genau dann positiv sein,
wenn die jeweiligen Winkel $\measuredangle BAX$, $\measuredangle XAC$,
$\measuredangle CBY$, $\measuredangle YBA$, $\measuredangle ACZ$ und
$\measuredangle ZCB$ im Uhrzeigersinn orientiert sind. (Siehe Fig.
\ref{fig.13}.)
\end{theorem}

In der Praxis hat niemand in einer Klausur die Zeit, die richtige Orientierung
nachzupr\"{u}fen; man macht sich an einer Figur klar, dass sie zumindest in
einem Fall stimmt, und argumentiert dann, dass der allgemeine Fall nicht
schlimmer sein kann (dahinter steckt die \textquotedblleft Permanenz von
polynomialen Identit\"{a}ten\textquotedblright).%

%TCIMACRO{\FRAME{ftbpFU}{4.7653in}{4.6662in}{0pt}{\Qcb{{}Satz von Ceva auf dem
%Kreis}}{\Qlb{fig.13}}{Figure}{\special{ language "Scientific Word";
%type "GRAPHIC";  maintain-aspect-ratio TRUE;  display "USEDEF";
%valid_file "T";  width 4.7653in;  height 4.6662in;  depth 0pt;
%original-width 4.6111in;  original-height 4.5146in;  cropleft "0";
%croptop "1";  cropright "1";  cropbottom "0";
%tempfilename 'KreisCeva1.wmf';tempfile-properties "XNPR";}}}%
%BeginExpansion
\begin{figure}[ptb]%
\centering
\includegraphics[
natheight=4.514600in,
natwidth=4.611100in,
height=4.6662in,
width=4.7653in
]%
{KreisCeva1.wmf}%
\caption{{}Satz von Ceva auf dem Kreis}%
\label{fig.13}%
\end{figure}
%EndExpansion


\textit{Beweis:} Sei $R$ der Umkreisradius des Dreiecks $ABC$. Dann ist $R$
auch der Umkreisradius des Dreiecks $ABX$. Nach dem erweiterten
Sinussatz\footnote{Der \textit{erweiterte Sinussatz} besagt folgendes: Ist $R$
der Umkreisradius eines Dreiecks $ABC$, dann gilt $BC=2R\sin\measuredangle
BAC$.} ist also%
\[
BX=2R\sin\measuredangle BAX.
\]
Und analog gilt%
\begin{align*}
XC  &  =2R\sin\measuredangle XAC;\ \ \ \ \ \ \ \ \ \ CY=2R\sin\measuredangle
CBY;\\
YA  &  =2R\sin\measuredangle YBA;\ \ \ \ \ \ \ \ \ \ AZ=2R\sin\measuredangle
ACZ;\\
ZB  &  =2R\sin\measuredangle ZCB.
\end{align*}
Somit gilt%
\begin{align*}
\dfrac{BX}{XC}\cdot\dfrac{CY}{YA}\cdot\dfrac{AZ}{ZB}  &  =\dfrac
{2R\sin\measuredangle BAX}{2R\sin\measuredangle XAC}\cdot\dfrac{2R\sin
\measuredangle CBY}{2R\sin\measuredangle YBA}\cdot\dfrac{2R\sin\measuredangle
ACZ}{2R\sin\measuredangle ZCB}\\
&  =\dfrac{\sin\measuredangle BAX}{\sin\measuredangle XAC}\cdot\dfrac
{\sin\measuredangle CBY}{\sin\measuredangle YBA}\cdot\dfrac{\sin\measuredangle
ACZ}{\sin\measuredangle ZCB}.
\end{align*}
Daher folgt Satz \ref{thm.kreisceva} aus Trig-Ceva.

\begin{exercise}
[isogonale Punkte]\label{exe.isogonal}Sei $ABC$ ein nicht-entartetes Dreieck.
Sei $P$ ein Punkt. Man beweise: Die Spiegelbilder der Geraden $AP$, $BP$ und
$CP$ an den Winkelhalbierenden der Winkel $CAB$, $ABC$ bzw. $BCA$ schneiden
sich in einem Punkt.
\end{exercise}

Letzterer Punkt hei\ss t der zu $P$ \emph{isogonal konjugierte} (oder kurz
\emph{isogonale}) Punkt bez\"{u}glich des Dreiecks $ABC$. (Siehe Fig.
\ref{fig.14}.) Zum Beispiel ist der H\"{o}henschnittpunkt zum
Umkreismittelpunkt isogonal (warum?). Die Inkreis- und Ankreismittelpunkte
sind jeweils zu sich selbst isogonal.%

%TCIMACRO{\FRAME{ftbpFU}{4.3985in}{5.2541in}{0pt}{\Qcb{Isogonale Punkte}%
%}{\Qlb{fig.14}}{Figure}{\special{ language "Scientific Word";
%type "GRAPHIC";  maintain-aspect-ratio TRUE;  display "USEDEF";
%valid_file "T";  width 4.3985in;  height 5.2541in;  depth 0pt;
%original-width 3.6979in;  original-height 4.4231in;  cropleft "0";
%croptop "1";  cropright "1";  cropbottom "0";
%tempfilename 'Isogonal1.wmf';tempfile-properties "XNPR";}}}%
%BeginExpansion
\begin{figure}[ptb]%
\centering
\includegraphics[
natheight=4.423100in,
natwidth=3.697900in,
height=5.2541in,
width=4.3985in
]%
{Isogonal1.wmf}%
\caption{Isogonale Punkte}%
\label{fig.14}%
\end{figure}
%EndExpansion


\textit{L\"{o}sung:} Seien $X$, $Y$ und $Z$ drei (jeweils von $A$, $B$ bzw.
$C$ verschiedene) Punkte auf diesen drei Spiegelbildern (siehe Fig.
\ref{fig.14b}). Dann \"{u}berf\"{u}hrt die Spiegelung an der
Winkelhalbierenden des Winkels $CAB$ die Gerade $AP$ in die Gerade $AX$,
w\"{a}hrend sie die Geraden $AB$ und $AC$ ineinander \"{u}berf\"{u}hrt. Somit
ist $\measuredangle BAX^{\prime}=\measuredangle PAC$ und $\measuredangle
X^{\prime}AC=\measuredangle BAP$, und analoge Gleichungen gelten f\"{u}r vier
andere Winkel. Somit ist%
\begin{align*}
&  \dfrac{\sin\measuredangle BAX^{\prime}}{\sin\measuredangle X^{\prime}%
AC}\cdot\dfrac{\sin\measuredangle CBY^{\prime}}{\sin\measuredangle Y^{\prime
}BA}\cdot\dfrac{\sin\measuredangle ACZ^{\prime}}{\sin\measuredangle Z^{\prime
}CB}\\
&  =\dfrac{\sin\measuredangle PAC}{\sin\measuredangle BAP}\cdot\dfrac
{\sin\measuredangle PBA}{\sin\measuredangle CBP}\cdot\dfrac{\sin\measuredangle
PCB}{\sin\measuredangle ACP}\\
&  =1\diagup\underbrace{\left(  \dfrac{\sin\measuredangle BAP}{\sin
\measuredangle PAC}\cdot\dfrac{\sin\measuredangle CBP}{\sin\measuredangle
PBA}\cdot\dfrac{\sin\measuredangle ACP}{\sin\measuredangle PCB}\right)
}_{\substack{=1\\\text{(nach Trig-Ceva, da die Geraden }AP\text{, }BP\text{
und }CP\\\text{sich in einem Punkt schneiden)}}}\\
&  =1\diagup1=1.
\end{align*}
Aus Trig-Ceva folgt nun wieder, dass sich die Geraden $AX$, $BY$ und $CZ$ in
einem Punkt schneiden. Diese Geraden sind aber genau die Spiegelbilder der
Geraden $AP$, $BP$ und $CP$ an den Winkelhalbierenden der Winkel $CAB$, $ABC$
bzw. $BCA$. Damit ist die Aufgabe gel\"{o}st.%

%TCIMACRO{\FRAME{ftbpFU}{4.2045in}{5.1177in}{0pt}{\Qcb{Beweis zu isogonalen
%Punkten}}{\Qlb{fig.14b}}{Figure}{\special{ language "Scientific Word";
%type "GRAPHIC";  maintain-aspect-ratio TRUE;  display "USEDEF";
%valid_file "T";  width 4.2045in;  height 5.1177in;  depth 0pt;
%original-width 3.6598in;  original-height 4.4612in;  cropleft "0";
%croptop "1";  cropright "1";  cropbottom "0";
%tempfilename 'Isogonal2.wmf';tempfile-properties "XNPR";}}}%
%BeginExpansion
\begin{figure}[ptb]%
\centering
\includegraphics[
natheight=4.461200in,
natwidth=3.659800in,
height=5.1177in,
width=4.2045in
]%
{Isogonal2.wmf}%
\caption{Beweis zu isogonalen Punkten}%
\label{fig.14b}%
\end{figure}
%EndExpansion


\begin{exercise}
Seien $X$, $Y$ und $Z$ die Fu\ss punkte der von $A$, $B$ bzw. $C$ ausgehenden
H\"{o}hen eines Dreiecks $ABC$. Auf dem Feuerbachkreis (d.h., auf dem Umkreis
des Dreiecks $XYZ$) seien $X^{\prime}$, $Y^{\prime}$ und $Z^{\prime}$ die zu
den Punkten $X$, $Y$ bzw. $Z$ diametral gegen\"{u}berliegenden Punkte. Man
zeige: Die Geraden $AX^{\prime}$, $BY^{\prime}$ und $CZ^{\prime}$ schneiden
sich in einem Punkt.
\end{exercise}

Dieser Punkt hei\ss t \emph{Prasolovpunkt} des Dreiecks $ABC$. (Siehe Fig.
\ref{fig.15}.)%

%TCIMACRO{\FRAME{ftbpFU}{5.0203in}{4.2934in}{0pt}{\Qcb{Der Prasolovpunkt}%
%}{\Qlb{fig.15}}{Figure}{\special{ language "Scientific Word";
%type "GRAPHIC";  maintain-aspect-ratio TRUE;  display "USEDEF";
%valid_file "T";  width 5.0203in;  height 4.2934in;  depth 0pt;
%original-width 4.4663in;  original-height 3.8156in;  cropleft "0";
%croptop "1";  cropright "1";  cropbottom "0";
%tempfilename 'Prasolov1.wmf';tempfile-properties "XNPR";}}}%
%BeginExpansion
\begin{figure}[ptb]%
\centering
\includegraphics[
natheight=3.815600in,
natwidth=4.466300in,
height=4.2934in,
width=5.0203in
]%
{Prasolov1.wmf}%
\caption{Der Prasolovpunkt}%
\label{fig.15}%
\end{figure}
%EndExpansion


\textit{L\"{o}sung:} Seien $\alpha$, $\beta$ und $\gamma$ die Winkel von
$\bigtriangleup ABC$ bei $A$, $B$, $C$.

Ein paar Bemerkungen vorneweg: Da die Punkte $Y^{\prime}$ und $Z^{\prime}$ auf
dem Feuerbachkreis jeweils den Punkten $Y$ bzw. $Z$ diametral
gegen\"{u}berliegen, sind $YY^{\prime}$ und $ZZ^{\prime}$ Durchmesser dieses
Feuerbachkreises. Somit sind alle Winkel im Viereck $YZ^{\prime}Y^{\prime}Z$
rechte Winkel (nach Thales). Daher ist $YZ^{\prime}Y^{\prime}Z$ ein Rechteck,
und daraus folgt $YZ^{\prime}=ZY^{\prime}$. Analog gilt $ZX^{\prime
}=XZ^{\prime}$ und $XY^{\prime}=YX^{\prime}$. Da $YZ^{\prime}Y^{\prime}Z$ ein
Rechteck ist, gilt ferner $\measuredangle ZYZ^{\prime}=90^{\circ}$. Analog ist
$\measuredangle XZX^{\prime}=90^{\circ}$.

Ferner ist bekannt, dass $\measuredangle BZX=\gamma$ ist. (Dies folgt aus dem
Sehnenviereckssatz, nachdem man bemerkt, dass die H\"{o}henfu\ss punkte $Z$
und $X$ beide auf dem Thaleskreis \"{u}ber $CA$ liegen.)%

%TCIMACRO{\FRAME{ftbpFU}{4.6518in}{4.0732in}{0pt}{\Qcb{Beweis zum
%Prasolovpunkt}}{\Qlb{fig.16}}{Figure}{\special{ language "Scientific Word";
%type "GRAPHIC";  maintain-aspect-ratio TRUE;  display "USEDEF";
%valid_file "T";  width 4.6518in;  height 4.0732in;  depth 0pt;
%original-width 4.0893in;  original-height 3.5784in;  cropleft "0";
%croptop "1";  cropright "1";  cropbottom "0";
%tempfilename 'Prasolov3.wmf';tempfile-properties "XNPR";}}}%
%BeginExpansion
\begin{figure}[ptb]%
\centering
\includegraphics[
natheight=3.578400in,
natwidth=4.089300in,
height=4.0732in,
width=4.6518in
]%
{Prasolov3.wmf}%
\caption{Beweis zum Prasolovpunkt}%
\label{fig.16}%
\end{figure}
%EndExpansion


Wir wollen nun Trig-Ceva anwenden; dazu brauchen wir $\dfrac{\sin
\measuredangle BAX^{\prime}}{\sin\measuredangle X^{\prime}AC}$. Wir machen
daf\"{u}r Sinusjagd (siehe Fig. \ref{fig.16}):%
\[
\sin\measuredangle BAX^{\prime}=\sin\measuredangle ZAX^{\prime}=ZX^{\prime
}\cdot\dfrac{\sin\measuredangle AZX^{\prime}}{AX^{\prime}}%
\ \ \ \ \ \ \ \ \ \ \left(  \text{Sinussatz im }\bigtriangleup ZAX^{\prime
}\right)
\]
und analog%
\[
\sin\measuredangle X^{\prime}AC=YX^{\prime}\cdot\dfrac{\sin\measuredangle
AYX^{\prime}}{AX^{\prime}}.
\]
Division ergibt%
\begin{align*}
\dfrac{\sin\measuredangle BAX^{\prime}}{\sin\measuredangle X^{\prime}AC}  &
=\dfrac{ZX^{\prime}\cdot\dfrac{\sin\measuredangle AZX^{\prime}}{AX^{\prime}}%
}{YX^{\prime}\cdot\dfrac{\sin\measuredangle AYX^{\prime}}{AX^{\prime}}}%
=\dfrac{ZX^{\prime}}{YX^{\prime}}\cdot\dfrac{\sin\measuredangle AZX^{\prime}%
}{\sin\measuredangle AYX^{\prime}}\\
&  =\dfrac{ZX^{\prime}}{YX^{\prime}}\cdot\dfrac{\sin\left(  90^{\circ}%
-\gamma\right)  }{\sin\left(  90^{\circ}-\beta\right)  },
\end{align*}
denn wir haben
\[
\measuredangle AZX^{\prime}=180^{\circ}-\underbrace{\measuredangle
XZX^{\prime}}_{=90^{\circ}}-\underbrace{\measuredangle BZX}_{=\gamma
}=180^{\circ}-90^{\circ}-\gamma=90^{\circ}-\gamma
\]
und analog $\measuredangle AYX^{\prime}=90^{\circ}-\beta$.

Multiplizieren wir diese Gleichung mit den zwei analogen, dann erhalten wir%
\begin{align*}
&  \dfrac{\sin\measuredangle BAX^{\prime}}{\sin\measuredangle X^{\prime}%
AC}\cdot\dfrac{\sin\measuredangle CBY^{\prime}}{\sin\measuredangle Y^{\prime
}BA}\cdot\dfrac{\sin\measuredangle ACZ^{\prime}}{\sin\measuredangle Z^{\prime
}CB}\\
&  =\left(  \dfrac{ZX^{\prime}}{YX^{\prime}}\cdot\dfrac{\sin\left(  90^{\circ
}-\gamma\right)  }{\sin\left(  90^{\circ}-\beta\right)  }\right)  \cdot\left(
\dfrac{XY^{\prime}}{ZY^{\prime}}\cdot\dfrac{\sin\left(  90^{\circ}%
-\alpha\right)  }{\sin\left(  90^{\circ}-\gamma\right)  }\right)  \cdot\left(
\dfrac{YZ^{\prime}}{XZ^{\prime}}\cdot\dfrac{\sin\left(  90^{\circ}%
-\beta\right)  }{\sin\left(  90^{\circ}-\gamma\right)  }\right) \\
&  =\dfrac{ZX^{\prime}}{YX^{\prime}}\cdot\dfrac{XY^{\prime}}{ZY^{\prime}}%
\cdot\dfrac{YZ^{\prime}}{XZ^{\prime}}=1\ \ \ \ \ \ \ \ \ \ \left(  \text{denn
}ZX^{\prime}=XZ^{\prime}\text{ und }XY^{\prime}=YX^{\prime}\text{ und
}YZ^{\prime}=ZY^{\prime}\right)
\end{align*}
Nach Trig-Ceva folgt hieraus die gew\"{u}nschte Aussage.

\textit{Alternativl\"{o}sung (Jurij) (skizziert):} Seien $\widetilde{X}$,
$\widetilde{Y}$ und $\widetilde{Z}$ die Bilder von $X$, $Y$ bzw. $Z$ bei der
Inversion am Umkreis des Dreiecks $ABC$. Sei $O$ der Mittelpunkt dieses
Umkreises. Nun zeigen wir:

\begin{itemize}
\item Die Geraden $A\widetilde{X}$, $B\widetilde{Y}$ und $C\widetilde{Z}$
schneiden sich in einem Punkt. (Dies ist \"{a}quivalent dazu, dass die Kreise
$AOX$, $BOY$ und $COZ$ sich in einem zweiten Punkt au\ss er $O$ schneiden.
Doch diesen Punkt kann man direkt beschreiben als den Punkt $O^{\prime}$ auf
der Geraden $OH$, der $HO\cdot HO^{\prime}=HA\cdot HX=HB\cdot HY=HC\cdot HZ$
erf\"{u}llt. Dass ein solcher Punkt $O^{\prime}$ existiert, ist bekannt.)

\item Die Geraden $AX^{\prime}$, $BY^{\prime}$ und $CZ^{\prime}$ sind die
Spiegelbilder der Geraden $A\widetilde{X}$, $B\widetilde{Y}$ und
$C\widetilde{Z}$ an den entsprechenden Winkelhalbierenden des Dreiecks $ABC$.
(Dies zeigt man anscheinend durch Winkeljagd.)
\end{itemize}

\begin{exercise}
[Satz von Haruki]Gegeben seien drei Kreise $k$, $\ell$ und $m$ mit insgesamt
sechs Schnittpunkten $A$, $B$, $C$, $D$, $E$ und $F$, in dem Sinne dass%
\[
\ell\cap m=\left\{  A,D\right\}  ,\ \ \ \ \ \ \ \ \ \ m\cap k=\left\{
B,E\right\}  \ \ \ \ \ \ \ \ \ \ \text{und}\ \ \ \ \ \ \ \ \ \ k\cap
\ell=\left\{  C,F\right\}
\]
ist. Zeige: $AE\cdot BF\cdot CD=EC\cdot FA\cdot DB$. (Siehe Fig. \ref{fig.17}.)
\end{exercise}

%

%TCIMACRO{\FRAME{ftbpFU}{4.889in}{5.1847in}{0pt}{\Qcb{{}Satz von Haruki}%
%}{\Qlb{fig.17}}{Figure}{\special{ language "Scientific Word";
%type "GRAPHIC";  maintain-aspect-ratio TRUE;  display "USEDEF";
%valid_file "T";  width 4.889in;  height 5.1847in;  depth 0pt;
%original-width 4.9644in;  original-height 5.2668in;  cropleft "0";
%croptop "1";  cropright "1";  cropbottom "0";
%tempfilename 'Haruki1.wmf';tempfile-properties "XNPR";}}}%
%BeginExpansion
\begin{figure}[ptb]%
\centering
\includegraphics[
natheight=5.266800in,
natwidth=4.964400in,
height=5.1847in,
width=4.889in
]%
{Haruki1.wmf}%
\caption{{}Satz von Haruki}%
\label{fig.17}%
\end{figure}
%EndExpansion


\begin{exercise}
\label{aufg.BZ-CY-AD}Sei $D$ ein Punkt auf der Seite $BC$ eines Dreiecks
$ABC$. Seien $Y$ und $Z$ die Inkreismittelpunkte der Dreiecke $ABD$ bzw.
$ACD$. Man beweise, dass sich die Geraden $BZ$, $CY$ und $AD$ genau dann in
einem Punkt schneiden, wenn $AD$ die Winkelhalbierende des Winkels $CAB$ ist.
(Siehe Fig. \ref{fig.18}.)
\end{exercise}

%

%TCIMACRO{\FRAME{ftbpFU}{4.7416in}{3.6479in}{0pt}{\Qcb{{}Zu Aufgabe
%\ref{aufg.BZ-CY-AD}}}{\Qlb{fig.18}}{Figure}%
%{\special{ language "Scientific Word";  type "GRAPHIC";
%maintain-aspect-ratio TRUE;  display "USEDEF";  valid_file "T";
%width 4.7416in;  height 3.6479in;  depth 0pt;  original-width 4.5883in;
%original-height 3.5242in;  cropleft "0";  croptop "1";  cropright "1";
%cropbottom "0";  tempfilename 'BZCYAD1.wmf';tempfile-properties "XNPR";}}}%
%BeginExpansion
\begin{figure}[ptb]%
\centering
\includegraphics[
natheight=3.524200in,
natwidth=4.588300in,
height=3.6479in,
width=4.7416in
]%
{BZCYAD1.wmf}%
\caption{{}Zu Aufgabe \ref{aufg.BZ-CY-AD}}%
\label{fig.18}%
\end{figure}
%EndExpansion


\begin{exercise}
\label{aufg.6-punkte-inversion-ceva}Seien $A$, $B$, $C$, $X$, $Y$ und $Z$
sechs paarweise verschiedene Punkte auf einem Kreis. Sei $k$ ein weiterer
Kreis. Seien $A^{\prime}$, $B^{\prime}$, $C^{\prime}$, $X^{\prime}$,
$Y^{\prime}$ und $Z^{\prime}$ die Bilder von $A$, $B$, $C$, $X$, $Y$ bzw. $Z$
bei der Inversion an $k$. Man zeige: Genau dann schneiden sich die Geraden
$AX$, $BY$ und $CZ$ in einem Punkt, wenn sich die Geraden $A^{\prime}%
X^{\prime}$, $B^{\prime}Y^{\prime}$ und $C^{\prime}Z^{\prime}$ in einem Punkt
schneiden. (Siehe Fig. \ref{fig.19}.)

[\textbf{Zusatz:} Man zeige in diesem Fall, dass diese beiden Schnittpunkte
mit dem Zentrum von $k$ auf einer Geraden liegen.]
\end{exercise}

%

%TCIMACRO{\FRAME{ftbpFU}{6.3885in}{6.2877in}{0pt}{\Qcb{{}Zu Aufgabe
%\ref{aufg.6-punkte-inversion-ceva}}}{\Qlb{fig.19}}{Figure}%
%{\special{ language "Scientific Word";  type "GRAPHIC";
%maintain-aspect-ratio TRUE;  display "USEDEF";  valid_file "T";
%width 6.3885in;  height 6.2877in;  depth 0pt;  original-width 6.0157in;
%original-height 5.9192in;  cropleft "0";  croptop "1";  cropright "1";
%cropbottom "0";  tempfilename 'Kreis6Punkte1.wmf';tempfile-properties "XNPR";}%
%}}%
%BeginExpansion
\begin{figure}[ptb]%
\centering
\includegraphics[
natheight=5.919200in,
natwidth=6.015700in,
height=6.2877in,
width=6.3885in
]%
{Kreis6Punkte1.wmf}%
\caption{{}Zu Aufgabe \ref{aufg.6-punkte-inversion-ceva}}%
\label{fig.19}%
\end{figure}
%EndExpansion


\subsection{Steiner und seine Folgerungen}

Mit dem Satz von Ceva (direkt angewendet auf ein gegebenes Dreieck $ABC$)
l\"{a}\ss t sich zeigen, dass die H\"{o}hen, die Seitenhalbierenden und die
Winkelhalbierenden eines Dreiecks $ABC$ konpunktal sind. F\"{u}r die
Mittelsenkrechten ist es aber nicht mehr so einfach, da sie keine
Ecktransversalen sind. Vergessen wir einmal, dass ihre Konpunktalit\"{a}t
ohnehin klar ist, und versuchen wir ein allgemeines Kriterium zu finden.

Der Satz von Steiner (aka Satz von Carnot) ist ein notwendiges und
hinreichendes Kriterium f\"{u}r die Konkurrenz von drei Geraden, die zu den
Seiten eines Dreiecks senkrecht sind:

\begin{theorem}
[Satz von Carnot-Steiner (bekannt als Satz von Steiner oder als Satz von
Carnot)]\label{thm.carnot-steiner}Sei $ABC$ ein nicht-entartetes Dreieck.
Seien $X$, $Y$ und $Z$ drei beliebige Punkte. Genau dann schneiden sich die
Senkrechten zu $BC$, $CA$ und $AB$ durch $X$, $Y$ bzw. $Z$ in einem Punkt,
wenn%
\[
BX^{2}-XC^{2}+CY^{2}-YA^{2}+AZ^{2}-ZB^{2}=0
\]
gilt. (Siehe Fig. \ref{fig.20} f\"{u}r eine Konfiguration, die diese zwei
Bedingungen \textbf{nicht} erf\"{u}llt.)
\end{theorem}

Hier ist es egal, ob die Strecken orientiert sind, denn sie werden ohnehin
quadriert. Daher kann man auch $CX^{2}$ statt $XC^{2}$ schreiben.%

%TCIMACRO{\FRAME{ftbpFU}{4.1283in}{3.625in}{0pt}{\Qcb{{}Satz von
%Carnot-Steiner}}{\Qlb{fig.20}}{Figure}{\special{ language "Scientific Word";
%type "GRAPHIC";  maintain-aspect-ratio TRUE;  display "USEDEF";
%valid_file "T";  width 4.1283in;  height 3.625in;  depth 0pt;
%original-width 4.3448in;  original-height 3.8116in;  cropleft "0";
%croptop "1";  cropright "1";  cropbottom "0";
%tempfilename 'Steiner2.wmf';tempfile-properties "XNPR";}}}%
%BeginExpansion
\begin{figure}[ptb]%
\centering
\includegraphics[
natheight=3.811600in,
natwidth=4.344800in,
height=3.625in,
width=4.1283in
]%
{Steiner2.wmf}%
\caption{{}Satz von Carnot-Steiner}%
\label{fig.20}%
\end{figure}
%EndExpansion


Zum Beweis ein Lemma:

\begin{lemma}
\label{lem.steiner-lem}Seien $X$, $Y$, $A$ und $B$ vier Punkte. Genau dann
gilt $XY\perp AB$, wenn $AX^{2}-AY^{2}=BX^{2}-BY^{2}$ gilt. (Hierbei tun wir
so, dass $XY\perp AB$ auch dann gilt, wenn $X=Y$ oder $A=B$ ist.)
\end{lemma}

\textit{Beweis von Lemma \ref{lem.steiner-lem}:} O. B. d. A. sei $X\neq Y$.
Seien $P$ und $Q$ die Fu\ss punkte der Lote von $A$ bzw. $B$ auf die Gerade
$XY$. (Siehe Fig. \ref{fig.21}.) Nach Pythagoras ist dann%
\[
AX^{2}=AP^{2}+PX^{2}\ \ \ \ \ \ \ \ \ \ \text{und}\ \ \ \ \ \ \ \ \ \ AY^{2}%
=AP^{2}+PY^{2}.
\]
Also ist (wobei wir orientierte Strecken auf der Geraden $XY$ verwenden)%
\begin{align*}
AX^{2}-AY^{2}  &  =\left(  AP^{2}+PX^{2}\right)  -\left(  AP^{2}%
+PY^{2}\right)  =PX^{2}-PY^{2}\\
&  =\underbrace{\left(  PX-PY\right)  }_{=YX}\cdot\underbrace{\left(
PX+PY\right)  }_{=2\cdot PX+XY}=YX\cdot\left(  2\cdot PX+XY\right)
\end{align*}
und analog%
\[
BX^{2}-BY^{2}=YX\cdot\left(  2\cdot QX+XY\right)  .
\]
%

%TCIMACRO{\FRAME{ftbpFU}{5.144in}{4.0977in}{0pt}{\Qcb{{}Zum Beweis des Lemmas}%
%}{\Qlb{fig.21}}{Figure}{\special{ language "Scientific Word";
%type "GRAPHIC";  maintain-aspect-ratio TRUE;  display "USEDEF";
%valid_file "T";  width 5.144in;  height 4.0977in;  depth 0pt;
%original-width 4.9796in;  original-height 3.9613in;  cropleft "0";
%croptop "1";  cropright "1";  cropbottom "0";
%tempfilename 'SteinerLem2.wmf';tempfile-properties "XNPR";}}}%
%BeginExpansion
\begin{figure}[ptb]%
\centering
\includegraphics[
natheight=3.961300in,
natwidth=4.979600in,
height=4.0977in,
width=5.144in
]%
{SteinerLem2.wmf}%
\caption{{}Zum Beweis des Lemmas}%
\label{fig.21}%
\end{figure}
%EndExpansion


Somit gilt folgende \"{A}quivalenzenkette:%
\begin{align*}
&  \ \left(  AX^{2}-AY^{2}=BX^{2}-BY^{2}\right) \\
&  \Longleftrightarrow\ \left(  YX\cdot\left(  2\cdot PX+XY\right)
=YX\cdot\left(  2\cdot QX+XY\right)  \right)  \ \ \ \ \ \ \ \ \ \ \left(
\text{nach obigen Gleichungen}\right) \\
&  \Longleftrightarrow\ \left(  2\cdot PX+XY=2\cdot QX+XY\right)
\ \ \ \ \ \ \ \ \ \ \left(  \text{denn }YX\neq0\right) \\
&  \Longleftrightarrow\ \left(  PX=QX\right) \\
&  \Longleftrightarrow\ \left(  P=Q\right)  \ \ \ \ \ \ \ \ \ \ \left(
\text{denn }P\text{ und }Q\text{ liegen auf }XY\text{, und die Strecken sind
orientiert}\right) \\
&  \Longleftrightarrow\ \left(  XY\perp AB\right)
\end{align*}
(denn $P$ und $Q$ sind die Lotfu\ss punkte von $A$ bzw. $B$ auf $XY$). Damit
ist das Lemma bewiesen.

Die Gleichung $AX^{2}-AY^{2}=BX^{2}-BY^{2}$ in Lemma \ref{lem.steiner-lem}
wird oft als $AX^{2}+BY^{2}=AY^{2}+BX^{2}$ geschrieben. Das Lemma l\"{a}\ss t
sich somit wie folgt umformulieren: \textquotedblleft Die Diagonalen eines
Vierecks $AXBY$ sind genau dann orthogonal, wenn die Summen der Quadrate
gegen\"{u}berliegenden Seiten des Vierecks gleich sind.\textquotedblright.

\textit{Beweis des Satzes von Carnot-Steiner:} Sei $P$ der Punkt, in dem die
Senkrechte zu $BC$ durch $X$ die Senkrechte zu $CA$ durch $Y$ schneidet. Dann
ist $XP\perp BC$ und $YP\perp CA$. Die Senkrechten zu $BC$, $CA$ und $AB$
durch $X$, $Y$ bzw. $Z$ schneiden sich genau dann in einem Punkt, wenn auch
die Senkrechte zu $AB$ durch $Z$ durch $P$ geht, d.h., wenn $ZP\perp AB$ ist.
Wegen $XP\perp BC$ folgt aus Lemma \ref{lem.steiner-lem} die Gleichung%
\[
BX^{2}-BP^{2}=CX^{2}-CP^{2}=XC^{2}-CP^{2},
\]
also%
\[
BX^{2}-XC^{2}=BP^{2}-CP^{2}.
\]
Analog folgt aus $YP\perp CA$ die Gleichung%
\[
CY^{2}-YA^{2}=CP^{2}-AP^{2}.
\]
Wenn nun auch $ZP\perp AB$ gilt, dann erhalten wir ebenso%
\[
AZ^{2}-BZ^{2}=AP^{2}-BP^{2},
\]
und durch Addition aller drei Gleichungen erhalten wir%
\begin{align*}
&  BX^{2}-XC^{2}+CY^{2}-YA^{2}+AZ^{2}-ZB^{2}\\
&  =BP^{2}-CP^{2}+CP^{2}-AP^{2}+AP^{2}-BP^{2}=0.
\end{align*}
Damit ist die $\Longrightarrow$-Richtung von Satz \ref{thm.carnot-steiner}
bewiesen. Die $\Longleftarrow$-Richtung folgt durch Umkrehrung dieses Argumentes.

Hier ist eine von vielen Anwendungen des Satzes von Carnot-Steiner:

\begin{theorem}
[Orthopolsatz]Sei $ABC$ ein Dreieck, und sei $g$ eine Gerade. Seien $X$, $Y$
und $Z$ die Fu\ss punkte der Lote von $A$, $B$ bzw. $C$ auf $g$. Dann
schneiden sich die Senkrechten zu $BC$, $CA$ und $AB$ durch $X$, $Y$ bzw. $Z$
in einem Punkt.
\end{theorem}

Dieser Punkt hei\ss t der \textit{Orthopol} der Geraden $g$ in bezug auf
Dreieck $ABC$. (Siehe Fig. \ref{fig.22}.)%

%TCIMACRO{\FRAME{ftbpFU}{5.0313in}{2.9321in}{0pt}{\Qcb{Der Orthopol einer
%Geraden $g$}}{\Qlb{fig.22}}{Figure}{\special{ language "Scientific Word";
%type "GRAPHIC";  maintain-aspect-ratio TRUE;  display "USEDEF";
%valid_file "T";  width 5.0313in;  height 2.9321in;  depth 0pt;
%original-width 4.6035in;  original-height 2.672in;  cropleft "0";
%croptop "1";  cropright "1";  cropbottom "0";
%tempfilename 'Orthopol1.wmf';tempfile-properties "XNPR";}}}%
%BeginExpansion
\begin{figure}[ptb]%
\centering
\includegraphics[
natheight=2.672000in,
natwidth=4.603500in,
height=2.9321in,
width=5.0313in
]%
{Orthopol1.wmf}%
\caption{Der Orthopol einer Geraden $g$}%
\label{fig.22}%
\end{figure}
%EndExpansion


\textit{Beweis:} Nach Pythagoras ist $BX^{2}=BY^{2}+XY^{2}$ und $XC^{2}%
=CZ^{2}+ZX^{2}$, also%
\[
BX^{2}-XC^{2}=\left(  BY^{2}+XY^{2}\right)  -\left(  CZ^{2}+ZX^{2}\right)  .
\]
Analog gelten \"{a}hnliche Formeln f\"{u}r $CY^{2}-YA^{2}$ und $AZ^{2}-ZB^{2}%
$. Wenn man alle drei Formeln zusammenaddiert, k\"{u}rzt sich alles heraus und
man erh\"{a}lt $0$. Nach Carnot-Steiner folgt hieraus die Behauptung.

Hier ist eine weitere Anwendung des Satzes von Carnot-Steiner:

\begin{theorem}
[Satz von den orthologischen Dreiecken]Seien $ABC$ und $XYZ$ zwei
nicht-entartete Dreiecke. Genau dann schneiden sich die Senkrechten zu $BC$,
$CA$ und $AB$ durch $X$, $Y$ bzw. $Z$ in einem Punkt, wenn die Senkrechten zu
$YZ$, $ZX$ und $XY$ durch $A$, $B$ bzw. $C$ sich in einem Punkt schneiden.
\end{theorem}

In diesem Fall hei\ss en die beiden Dreiecke $ABC$ und $XYZ$ zueinander
\textit{orthologisch}. (Siehe Fig. \ref{fig.23}.)%

%TCIMACRO{\FRAME{ftbpFU}{4.845in}{3.6208in}{0pt}{\Qcb{Orthologische Dreiecke}%
%}{\Qlb{fig.23}}{Figure}{\special{ language "Scientific Word";
%type "GRAPHIC";  maintain-aspect-ratio TRUE;  display "USEDEF";
%valid_file "T";  width 4.845in;  height 3.6208in;  depth 0pt;
%original-width 4.4586in;  original-height 3.3251in;  cropleft "0";
%croptop "1";  cropright "1";  cropbottom "0";
%tempfilename 'Ortholog1.wmf';tempfile-properties "XNPR";}}}%
%BeginExpansion
\begin{figure}[ptb]%
\centering
\includegraphics[
natheight=3.325100in,
natwidth=4.458600in,
height=3.6208in,
width=4.845in
]%
{Ortholog1.wmf}%
\caption{Orthologische Dreiecke}%
\label{fig.23}%
\end{figure}
%EndExpansion


\textit{Beweis:} Laut dem Satz von Carnot-Steiner gilt:

\begin{itemize}
\item Genau dann schneiden sich die Senkrechten zu $BC$, $CA$ und $AB$ durch
$X$, $Y$ bzw. $Z$ in einem Punkt, wenn die Gleichung%
\[
BX^{2}-XC^{2}+CY^{2}-YA^{2}+AZ^{2}-ZB^{2}=0
\]
gilt.

\item Genau dann schneiden sich die Senkrechten zu $YZ$, $ZX$ und $XY$ durch
$A$, $B$ bzw. $C$ in einem Punkt, wenn die Gleichung%
\[
YA^{2}-AZ^{2}+ZB^{2}-BX^{2}+XC^{2}-CY^{2}=0
\]
gilt.
\end{itemize}

Aber die beiden Gleichungen sind zueinander \"{a}quivalent, da die linke Seite
der einen gleich der linken Seite der anderen mal $-1$ ist. Hieraus folgt der Satz.

\textit{Warnung:} Der Satz von den orthologischen Dreiecken (wie auch der Satz
von Carnot-Steiner) gilt nicht f\"{u}r entartete Dreiecke.

K\"{o}nnen wir den Orthopolsatz aus dem Satz von den orthologischen Dreiecken
herleiten? Das Dreieck $XYZ$ im Orthopolsatz ist entartet; somit k\"{o}nnen
wir den Satz von den orthologischen Dreiecken nicht auf dieses Dreieck anwenden.

Allerdings k\"{o}nnen wir folgendes tun:

Wir spiegeln die Ecken $A$, $B$ und $C$ an der Geraden $g$. Die Spiegelbilder
seien $X^{\prime}$, $Y^{\prime}$ bzw. $Z^{\prime}$ (siehe Fig. \ref{fig.24}).
Nun werden wir

\begin{enumerate}
\item zeigen, dass die Dreiecke $ABC$ und $X^{\prime}Y^{\prime}Z^{\prime}$
orthologisch sind, und dann

\item den Orthopolsatz daraus herleiten.
\end{enumerate}

%

%TCIMACRO{\FRAME{ftbpFU}{6.7155in}{5.5888in}{0pt}{\Qcb{Zweiter Beweis des
%Orthopolsatzes}}{\Qlb{fig.24}}{Figure}{\special{ language "Scientific Word";
%type "GRAPHIC";  maintain-aspect-ratio TRUE;  display "USEDEF";
%valid_file "T";  width 6.7155in;  height 5.5888in;  depth 0pt;
%original-width 6.5833in;  original-height 5.4744in;  cropleft "0";
%croptop "1";  cropright "1";  cropbottom "0";
%tempfilename 'Orthopol2.wmf';tempfile-properties "XNPR";}}}%
%BeginExpansion
\begin{figure}[ptb]%
\centering
\includegraphics[
natheight=5.474400in,
natwidth=6.583300in,
height=5.5888in,
width=6.7155in
]%
{Orthopol2.wmf}%
\caption{Zweiter Beweis des Orthopolsatzes}%
\label{fig.24}%
\end{figure}
%EndExpansion


F\"{u}r Schritt 1 verwenden wir folgenden Satz:

\begin{theorem}
[Orthologie gegensinnig \"{a}hnlicher Dreiecke]\label{thm.ortholog-gegen}Seien
$ABC$ und $XYZ$ zwei Dreiecke, die zueinander gegensinnig \"{a}hnlich sind.
Dann sind diese zwei Dreiecke zueinander orthologisch. Mehr sogar:

\textbf{(a)} Die Senkrechten zu $BC$, $CA$ und $AB$ durch $X$, $Y$ bzw. $Z$
schneiden sich in einem Punkt auf dem Umkreis des Dreiecks $XYZ$. (Siehe Fig.
\ref{fig.25}.)

\textbf{(b)} Die Senkrechten zu $YZ$, $ZX$ und $XY$ durch $A$, $B$ bzw. $C$
schneiden sich in einem Punkt auf dem Umkreis des Dreiecks $ABC$.
\end{theorem}

%

%TCIMACRO{\FRAME{ftbpFU}{4.7052in}{4.2892in}{0pt}{\Qcb{Gegensinnig \"{a}hnliche
%Dreiecke sind orthologisch}}{\Qlb{fig.25}}{Figure}%
%{\special{ language "Scientific Word";  type "GRAPHIC";
%maintain-aspect-ratio TRUE;  display "USEDEF";  valid_file "T";
%width 4.7052in;  height 4.2892in;  depth 0pt;  original-width 4.2053in;
%original-height 3.8309in;  cropleft "0";  croptop "1";  cropright "1";
%cropbottom "0";
%tempfilename 'GegensinnigOrtholog1.wmf';tempfile-properties "XNPR";}}}%
%BeginExpansion
\begin{figure}[ptb]%
\centering
\includegraphics[
natheight=3.830900in,
natwidth=4.205300in,
height=4.2892in,
width=4.7052in
]%
{GegensinnigOrtholog1.wmf}%
\caption{Gegensinnig \"{a}hnliche Dreiecke sind orthologisch}%
\label{fig.25}%
\end{figure}
%EndExpansion


\textit{Beweis:} Winkeljagd!

\textbf{(a)} (Siehe Fig. \ref{fig.26}.) Sei $P$ der von $X$ verschiedene
Punkt, in dem die Senkrechte zu $BC$ durch $X$ den Umkreis des Dreiecks $XYZ$
schneidet. (Wenn sie diesen Umkreis ber\"{u}hrt, dann setzt man $P=X$.) Mit
orientierten Winkeln modulo $180^{\circ}$ gilt%
\begin{align*}
\measuredangle\left(  XP;\ YP\right)   &  =\measuredangle XPY=\measuredangle
XZY\ \ \ \ \ \ \ \ \ \ \left(  \text{nach dem Umfangswinkelsatz}\right) \\
&  =-\measuredangle ACB\ \ \ \ \ \ \ \ \ \ \left(  \text{da }\bigtriangleup
XYZ\overset{-}{\sim}\Delta ABC\right) \\
&  =-\measuredangle\left(  CA;\ BC\right)  =\measuredangle\left(
BC;\ CA\right) \\
&  =\underbrace{\measuredangle\left(  BC;\ XP\right)  }_{\substack{=90^{\circ
}\\\text{(denn }XP\perp BC\text{)}}}+\measuredangle\left(  XP;\ CA\right) \\
&  \ \ \ \ \ \ \ \ \ \ \ \ \ \ \ \ \ \ \ \ \left(
\begin{array}
[c]{c}%
\text{denn }\measuredangle\left(  g;\ k\right)  =\measuredangle\left(
g;\ h\right)  +\measuredangle\left(  h;\ k\right)  \text{ gilt}\\
\text{f\"{u}r beliebige drei Geraden }g\text{, }h\text{ und }k
\end{array}
\right) \\
&  =90^{\circ}+\measuredangle\left(  XP;\ CA\right)  .
\end{align*}
Daher ist%
\[
90^{\circ}=\measuredangle\left(  XP;\ YP\right)  -\measuredangle\left(
XP;\ CA\right)  =\measuredangle\left(  CA;\ YP\right)
\]
(denn $\measuredangle\left(  g;\ k\right)  -\measuredangle\left(
g;\ h\right)  =\measuredangle\left(  h;\ k\right)  $ gilt f\"{u}r beliebige
drei Geraden $g$, $h$ und $k$). Hieraus folgt $YP\perp CA$, und daher liegt
$P$ auch auf der Senkrechten zu $CA$ durch $Y$. Analog liegt $P$ auf der
Senkrechten zu $AB$ durch $Z$. Also schneiden sich die drei Senkrechten zu
$BC$, $CA$ und $AB$ durch $X$, $Y$ bzw. $Z$ in einem Punkt auf dem Umkreis des
Dreiecks $XYZ$ (n\"{a}mlich in $P$). Damit ist Teil \textbf{(a)} bewiesen.%

%TCIMACRO{\FRAME{ftbpFU}{4.6459in}{4.1748in}{0pt}{\Qcb{Zum Beweis}%
%}{\Qlb{fig.26}}{Figure}{\special{ language "Scientific Word";
%type "GRAPHIC";  maintain-aspect-ratio TRUE;  display "USEDEF";
%valid_file "T";  width 4.6459in;  height 4.1748in;  depth 0pt;
%original-width 4.2587in;  original-height 3.8233in;  cropleft "0";
%croptop "1";  cropright "1";  cropbottom "0";
%tempfilename 'GegensinnigOrtholog2.wmf';tempfile-properties "XNPR";}}}%
%BeginExpansion
\begin{figure}[ptb]%
\centering
\includegraphics[
natheight=3.823300in,
natwidth=4.258700in,
height=4.1748in,
width=4.6459in
]%
{GegensinnigOrtholog2.wmf}%
\caption{Zum Beweis}%
\label{fig.26}%
\end{figure}
%EndExpansion


\textbf{(b)} folgt aus Teil \textbf{(a)}, wenn man die Rollen der zwei
Dreiecke $ABC$ und $XYZ$ vertauscht.

\textit{Zweiter Beweis des Orthopolsatzes (skizziert):} Seien $X^{\prime}$,
$Y^{\prime}$ und $Z^{\prime}$ die Spiegelbilder von $X$, $Y$ bzw. $Z$ an der
Geraden $g$. Dann sind die Lotfu\ss punkte $X$, $Y$ und $Z$ die Mittelpunkte
der Strecken $AX^{\prime}$, $BY^{\prime}$ bzw. $CZ^{\prime}$.

Aber da $X^{\prime}$, $Y^{\prime}$ und $Z^{\prime}$ die Spiegelbilder von $A$,
$B$ bzw. $C$ an der Geraden $g$ sind, sind die Dreiecke $ABC$ und $X^{\prime
}Y^{\prime}Z^{\prime}$ gegensinnig \"{a}hnlich (sogar kongruent), und daher
laut Satz \ref{thm.ortholog-gegen} orthologisch. Die Senkrechten von
$X^{\prime}$, $Y^{\prime}$ und $Z^{\prime}$ auf $BC$, $CA$ bzw. $AB$ schneiden
sich also in einem Punkt $Q$. Die Senkrechten von $A$, $B$ und $C$ auf $BC$,
$CA$ bzw. $AB$ schneiden sich aber auch in einem Punkt, n\"{a}mlich im
H\"{o}henschnittpunkt $H$ des Dreiecks $ABC$. Hieraus folgt schnell, dass die
Senkrechten von $X$, $Y$ und $Z$ auf $BC$, $CA$ bzw. $AB$ sich im Mittelpunkt
von $HQ$ schneiden (denn $X$ ist ja der Mittelpunkt von $AX^{\prime}$ usw.,
und somit hat man Mittelparallelen in Trapezen vorliegen). Damit ist der
Orthopolsatz wieder gezeigt.

\begin{exercise}
Sei $ABC$ ein Dreieck, und sei $g$ eine Gerade durch den Umkreismittelpunkt
des Dreiecks $ABC$. Sei $P$ der Orthopol der Geraden $g$ in bezug auf das
Dreieck $ABC$. Sei $H$ der H\"{o}henschnittpunkt des Dreiecks $ABC$. Man
zeige: Das Spiegelbild von $H$ an $P$ liegt auf dem Umkreis des Dreiecks
$ABC$. (Siehe Fig. \ref{fig.27}.)
\end{exercise}

%

%TCIMACRO{\FRAME{ftbpFU}{4.8551in}{4.2409in}{0pt}{\Qcb{Orthopol einer Geraden
%durch den Umkreismittelpunkt}}{\Qlb{fig.27}}{Figure}%
%{\special{ language "Scientific Word";  type "GRAPHIC";
%maintain-aspect-ratio TRUE;  display "USEDEF";  valid_file "T";
%width 4.8551in;  height 4.2409in;  depth 0pt;  original-width 4.2587in;
%original-height 3.7157in;  cropleft "0";  croptop "1";  cropright "1";
%cropbottom "0";
%tempfilename 'OrthopolUmkreis1.wmf';tempfile-properties "XNPR";}}}%
%BeginExpansion
\begin{figure}[ptb]%
\centering
\includegraphics[
natheight=3.715700in,
natwidth=4.258700in,
height=4.2409in,
width=4.8551in
]%
{OrthopolUmkreis1.wmf}%
\caption{Orthopol einer Geraden durch den Umkreismittelpunkt}%
\label{fig.27}%
\end{figure}
%EndExpansion


Hier ist eine weitere Anwendung des Satzes von den orthologischen Dreiecken:

\begin{exercise}
Sei $ABC$ ein Dreieck. Sei $P$ ein Punkt. Seien $X$, $Y$ und $Z$ die
Fu\ss punkte der Lote von $P$ auf $BC$, $CA$ bzw. $AB$. Man zeige:

\textbf{(a)} Die Senkrechten zu $YZ$, $ZX$ und $XY$ durch $A$, $B$ bzw. $C$
schneiden sich in einem Punkt.

\textbf{(b)} Diese Senkrechten sind genau die Spiegelbilder der Geraden $AP$,
$BP$ bzw. $CP$ an den Winkelhalbierenden der Winkel $CAB$, $ABC$ bzw. $BCA$.

\textbf{(c)} Man beweise den Satz vom isogonalen Punkt neu.
\end{exercise}

\begin{exercise}
\label{aufg.AS-MN}Sei $ABC$ ein Dreieck, und seien $E$ und $F$ die
Fu\ss punkte seiner von $B$ bzw. $C$ ausgehenden H\"{o}hen. Seien $M$, $S$ und
$N$ die Mittelpunkte der Strecken $BF$, $EF$ bzw. $CE$. Man zeige:

\textbf{(a)} Es gilt $AS\perp MN$.

\textbf{(b)} Die Senkrechte zu $BS$ durch $M$ und die Senkrechte zu $CS$ durch
$N$ schneiden sich auf der Mittelsenkrechten von $BC$. (Siehe Fig.
\ref{fig.28}.)
\end{exercise}

%

%TCIMACRO{\FRAME{ftbpFU}{4.4104in}{3.7436in}{0pt}{\Qcb{{}Zu Aufgabe
%\ref{aufg.AS-MN}}}{\Qlb{fig.28}}{Figure}%
%{\special{ language "Scientific Word";  type "GRAPHIC";
%maintain-aspect-ratio TRUE;  display "USEDEF";  valid_file "T";
%width 4.4104in;  height 3.7436in;  depth 0pt;  original-width 4.2663in;
%original-height 3.6157in;  cropleft "0";  croptop "1";  cropright "1";
%cropbottom "0";  tempfilename 'ASMN1.wmf';tempfile-properties "XNPR";}}}%
%BeginExpansion
\begin{figure}[ptb]%
\centering
\includegraphics[
natheight=3.615700in,
natwidth=4.266300in,
height=3.7436in,
width=4.4104in
]%
{ASMN1.wmf}%
\caption{{}Zu Aufgabe \ref{aufg.AS-MN}}%
\label{fig.28}%
\end{figure}
%EndExpansion


\begin{exercise}
Man beweise, dass der Satz von den orthologischen Dreiecken auch dann gilt,
wenn man alle \textquotedblleft Senkrechten\textquotedblright\ durch
\textquotedblleft Parallelen\textquotedblright\ ersetzt. Das hei\ss t:
\end{exercise}

\begin{theorem}
Seien $ABC$ und $XYZ$ zwei nicht-entartete Dreiecke. Genau dann schneiden sich
die Parallelen zu $BC$, $CA$ und $AB$ durch $X$, $Y$ bzw. $Z$ in einem Punkt,
wenn die Parallelen zu $YZ$, $ZX$ und $XY$ durch $A$, $B$ bzw. $C$ sich in
einem Punkt schneiden.
\end{theorem}

In diesem Fall hei\ss en die Dreiecke $ABC$ und $XYZ$ zueinander
\emph{parallelologisch}.

\begin{exercise}
\label{aufg.gerg-nag-ortho}Sei $ABC$ ein Dreieck. Der Inkreis des Dreiecks
$ABC$ ber\"{u}hre die Strecken $BC$, $CA$ und $AB$ in den Punkten $X$, $Y$
bzw. $Z$. Der $A$-Ankreis des Dreiecks $ABC$ (also der Ankreis, der die
Strecke $BC$ ber\"{u}hrt) ber\"{u}hre die Strecke $BC$ in $X^{\prime}$. Sei
$x$ die Senkrechte von $X^{\prime}$ auf die Gerade $YZ$. Analog seien zwei
Geraden $y$ und $z$ definiert. Man zeige, dass $x$, $y$ und $z$ sich in einem
Punkt schneiden. (Siehe Fig. \ref{fig.29}.)
\end{exercise}

%

%TCIMACRO{\FRAME{ftbpFU}{6.9595in}{5.1897in}{0pt}{\Qcb{{}Zu Aufgabe
%\ref{aufg.gerg-nag-ortho}}}{\Qlb{fig.29}}{Figure}%
%{\special{ language "Scientific Word";  type "GRAPHIC";
%maintain-aspect-ratio TRUE;  display "USEDEF";  valid_file "T";
%width 6.9595in;  height 5.1897in;  depth 0pt;  original-width 6.4224in;
%original-height 4.7831in;  cropleft "0";  croptop "1";  cropright "1";
%cropbottom "0";
%tempfilename 'GergNagelOrtholog1.wmf';tempfile-properties "XNPR";}}}%
%BeginExpansion
\begin{figure}[ptb]%
\centering
\includegraphics[
natheight=4.783100in,
natwidth=6.422400in,
height=5.1897in,
width=6.9595in
]%
{GergNagelOrtholog1.wmf}%
\caption{{}Zu Aufgabe \ref{aufg.gerg-nag-ortho}}%
\label{fig.29}%
\end{figure}
%EndExpansion


Die nachfolgende Aufgabe ergibt einen neuen Beweis des Satzes von den
orthologischen Dreiecken, bei dem auch eine andere sehr n\"{u}tzliche
Konfiguration hervortritt (eine verallgemeinerte Fermatkonfiguration, f\"{u}r
alle denen der Name etwas sagt). Zun\"{a}chst einige Bezeichnungen:

Wenn wir schreiben, dass zwei Dreiecke $ABC$ und $XYZ$ \"{a}hnlich sind,
meinen wir immer, dass es eine \"{A}hnlichkeitsabbildung gibt, die $A$ auf $X$
abbildet, $B$ auf $Y$ abbildet, und $C$ auf $Z$ abbildet. Entsprechende Ecken
stehen also immer an der gleichen Stelle. Es gibt also einen Unterschied
zwischen \textquotedblleft die Dreiecke $ABC$ und $XYZ$ sind
\"{a}hnlich\textquotedblright\ und \textquotedblleft die Dreiecke $ABC$ und
$YZX$ sind \"{a}hnlich\textquotedblright. Ferner nennen wir zwei \"{a}hnliche
Dreiecke $ABC$ und $XYZ$ \textit{gleichsinnig \"{a}hnlich} oder
\textit{gegensinnig \"{a}hnlich}, je nachdem, ob die
\"{A}hnlichkeitsabbildung, die $A$, $B$ und $C$ auf $X$, $Y$ bzw. $Z$
abbildet, gleichsinnig oder gegensinnig ist.

Wenn $U$, $V$ und $W$ drei Punkte sind, dann bezeichnen wir den Umkreis des
Dreiecks $UVW$ auch kurz als den \textquotedblleft Kreis $UVW$%
\textquotedblright. Falls $U$, $V$ und $W$ auf einer Geraden liegen, wird
diese Gerade als \textquotedblleft Kreis $UVW$\textquotedblright\ verstanden.%

%TCIMACRO{\FRAME{ftbpFU}{6.4681in}{6.1225in}{0pt}{\Qcb{{}Zu Aufgabe
%\ref{aufg.double-staerk}}}{\Qlb{fig.30}}{Figure}%
%{\special{ language "Scientific Word";  type "GRAPHIC";
%maintain-aspect-ratio TRUE;  display "USEDEF";  valid_file "T";
%width 6.4681in;  height 6.1225in;  depth 0pt;  original-width 6.6452in;
%original-height 6.2885in;  cropleft "0";  croptop "1";  cropright "1";
%cropbottom "0";  tempfilename 'Staerk1.wmf';tempfile-properties "XNPR";}}}%
%BeginExpansion
\begin{figure}[ptb]%
\centering
\includegraphics[
natheight=6.288500in,
natwidth=6.645200in,
height=6.1225in,
width=6.4681in
]%
{Staerk1.wmf}%
\caption{{}Zu Aufgabe \ref{aufg.double-staerk}}%
\label{fig.30}%
\end{figure}
%EndExpansion


\begin{exercise}
\label{aufg.double-staerk}Seien $ABC$ und $XYZ$ zwei Dreiecke.

Seien drei Punkte $A^{\prime}$, $B^{\prime}$ und $C^{\prime}$ so gew\"{a}hlt,
dass die Dreiecke $A^{\prime}BC$, $AB^{\prime}C$ und $ABC^{\prime}$ jeweils
zum Dreieck $XYZ$ gegensinnig \"{a}hnlich sind.

\textbf{(a)} Man zeige: Die Kreise $A^{\prime}BC$, $AB^{\prime}C$ und
$ABC^{\prime}$ und die Geraden $AA^{\prime}$, $BB^{\prime}$ und $CC^{\prime}$
schneiden sich in einem Punkt $P$. (Siehe Fig. \ref{fig.30}.)

Seien drei Punkte $X^{\prime}$, $Y^{\prime}$ und $Z^{\prime}$ so gew\"{a}hlt,
dass die Dreiecke $X^{\prime}YZ$, $XY^{\prime}Z$ und $XYZ^{\prime}$ jeweils
zum Dreieck $ABC$ gegensinnig \"{a}hnlich sind.

\textbf{(b)} Man zeige: Die Kreise $X^{\prime}YZ$, $XY^{\prime}Z$ und
$XYZ^{\prime}$ und die Geraden $XX^{\prime}$, $YY^{\prime}$ und $ZZ^{\prime}$
schneiden sich in einem Punkt $Q$.

\textbf{(c)} Man zeige: $\measuredangle\left(  AA^{\prime};\ YZ\right)
=-\measuredangle\left(  XX^{\prime};\ BC\right)  $. Hierbei ist
$\measuredangle\left(  g;\ h\right)  $ der orientierte Winkel zwischen zwei
Geraden $g$ und $h$ (ein orientierter Winkel modulo $180^{\circ}$).

\textbf{(d)} Man zeige: Genau dann ist $AA^{\prime}\perp YZ$, wenn
$XX^{\prime}\perp BC$ ist.

\textbf{(e)} Man folgere den Satz von den orthologischen Dreiecken.

Jetzt nehmen wir an, dass das Dreieck $XYZ$ gleichseitig und zum Dreieck $ABC$
gleich orientiert ist. Die Dreiecke $A^{\prime}BC$, $AB^{\prime}C$ und
$ABC^{\prime}$ sind also gleichseitige Dreiecke, die auf den Seiten $BC$, $CA$
bzw. $AB$ des Dreiecks $ABC$ nach au\ss en hin aufgesetzt wurden. Der Punkt
$P$ hei\ss t dann der \emph{erste Fermatpunkt} des Dreiecks $ABC$. Unter
diesen Behauptungen zeige man folgendes:

\textbf{(f)} Es gilt $AA^{\prime}=BB^{\prime}=CC^{\prime}=AP+BP+CP$.

\textbf{(g)} Wir nehmen an, dass alle Winkel des Dreiecks $ABC$ kleiner oder
gleich $120^{\circ}$ sind. Dann ist $P$ derjenige Punkt in der Ebene, f\"{u}r
den die Summe der Abst\"{a}nde zu $A$, $B$ und $C$ minimal ist; das hei\ss t,
wir haben%
\[
AP+BP+CP\leq AR+BR+CR
\]
f\"{u}r alle Punkte $R$.
\end{exercise}


\end{document}